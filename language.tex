\section{Language}
\label{sec:language}

\begin{figure}
\small
\hrulefill
\begin{align*}
  \text{value}~v
    \bnfis {} &x & & \text{variable} \\
    \bnfor {} &\tru \bnfor \fls & & \text{boolean constants} \\
    \bnfor {} &\fun{x} c & & \text{function} \\
    \bnfor {} &h & & \text{handler} \\
  \text{handler}~h
    \bnfis {} &\handler \{
      \retclause{x} c_r, & & \text{(optional) return clause} \\
      &\hphantom{\handler \{ {}}
      \opclause[\op_1]{x}{k} c_1,
      \dots,
      \opclause[\op_n]{x}{k} c_n \} & & \text{operation clauses} \\
  \text{computation}~c
    \bnfis {} &\return v & & \text{return} \\
    \bnfor {} &\opcall{v}{y}{c} & & \text{operation call} \\
    \bnfor {} &\seq{x}{c_1} c_2 & & \text{sequencing} \\
    \bnfor {} &\conditional{v}{c_1}{c_2} & & \text{conditional} \\
    \bnfor {} &v_1 \, v_2 & & \text{application} \\
    \bnfor {} &\withhandle{v}{c} & & \text{handling}
\end{align*}
\hrulefill
\caption{Syntax of terms.}
\label{fig:terms}  
\end{figure}

Before we dive into examples of handlers, we need to fix a language in which to
work. As the order of evaluation is important when dealing with effects, we
split language terms (Figure~\ref{fig:terms}) into inert \emph{values} and
potentially effectful \emph{computations}, following an approach called
\emph{fine-grain call-by-value}~\cite{DBLP:journals/iandc/LevyPT03}. There are a few
things worth mentioning:
%
\begin{description}
\item[Sequencing]
In $\seq{x}{c_1}{c_2}$, we first evaluate $c_1$, and once this returns a value,
we bind it to $x$ and proceed by $c_2$. If $x$ does not appear in $c_2$,
we abbreviate the sequencing to $c_1; c_2$.
\item[Operation calls]
The call~$\opcall{v}{y}{c}$ passes a \emph{parameter} value~$v$
(e.g.\ the memory location to be read) to the operation $\op$, and after
$\op$ performs the effect, its \emph{result} value (e.g.\ the contents of the memory
location) is bound to~$y$ and the evaluation of $c$, called a \emph{continuation},
resumes. However, note that encompassing handlers may override this behaviour.
\item[Generic effects]
Having an explicit continuation in the call is convenient for the semantics, but
less so for a programmer, who just wants to get back the result of an
operation. So, instead of a full-blown operation call, we define a function,
called a \emph{generic effect}~\cite{DBLP:journals/acs/PlotkinP03}, also labelled
as $\opgen$, which takes a parameter and passes it to an operation call
with the trivial continuation:
\[
  \opgen \defeq \fun{x} \opcall{x}{y}{\return y}
\]
Though simpler to use, generic effects are just as expressive because we can
recover the operation call $\opcall{v}{y}{c}$ by evaluating $\seq{y}{\opgen \, v} c$.
\item[Language extensions]
To focus on new constructs, we shall keep our language small, but for examples,
we are going to extend its values with integers, primitive arithmetic functions,
strings, recursive functions $\recfun{f}{x} c$, the unit $\unt$ and pairs $(v_1,
v_2)$. Furthermore, we allow patterns in binding constructs (functions, handler
clauses, operation calls, and sequencing). In particular, we use the pattern
$\anon$ to denote ignored parameters, and a pair pattern $(x_1, x_2)$ to extract
components from a pair. For example, we bind $7$ to $x$ and ignore $8$ in the
application $(\fun{(x, \anon)} 6 + x) \, (7, 8)$.
\item[Separation of values \& computations]
We were a bit lax about the separation of values and computations when writing
the last example. Since the addition $6 + x$ is in fact a double application
$((+) \, 6) \, x$, the first application $(+) \, 6$ is already a
computation. Thus, it cannot be applied to $x$ because both subterms of an application
must be values. Instead, we need to use sequencing and write the example in our restricted syntax as:
\[
  (\fun{(x, \anon)} \seq{f}{(+) \, 6} f \, x) \, (7, 8)    
\]
However, this longer form adds little value and makes examples hard to read, so
while keeping it in mind, we are going to use the shorter form from now on.

Conversely, we shall implicitly insert $\kord{return}$ whenever we use a value
where a computation is expected. For example, we shall write
${\fun{x} \fun{y} (x, y)}$ instead of
$\fun{x} \return (\fun{y} \return (x, y))$.

\item[Semantics]
Observe that each operation call
creates a branching point in the evaluation, with as many branches as there are
possible results that can be yielded to the continuation. For example, $\decide$
will have two branches, $\print$ just one, and $\read$ will have infinite many
branches: one for each possible input. Thus, we can imagine computations as
trees, whose leaves are returned values and branching points are called operations. For an
example, see Figure~\ref{fig:tree}.

\begin{figure}[h]
\small
\hrulefill
\[
  \begin{aligned}
    &\print \, \str{A}; \\
    &\seq{n}{\get \, \unt} \\
    &\conditional{n < 0}{\\&\quad\print \, \str{B};\\&\quad \return{-n^2}\\&}{\\&\quad \return{n + 1}}
  \end{aligned}
  \qquad
  \begin{tikzpicture}[level distance=1cm,baseline=(current bounding box.east)]
    \node {$\print \, \str{A}$}
      child {
        node {$\get \, \unt$}
          child {
            node {$\cdots$}
          }
          child {
            node {$\print \, \str{B}$}
              child {
                node {$-4$}
              }
            edge from parent node[fill=white, inner sep=0pt] {\tiny $-2$}
          }
          child {
            node {$\print \, \str{B}$}
              child {
                node {$-1$}
              }
            edge from parent node[fill=white, inner sep=1pt] {\tiny $-1$}
          }
          child {
            node {$1$}
            edge from parent node[fill=white, inner sep=1pt] {\tiny $0$}
          }
          child {
            node {$2$}
            edge from parent node[fill=white, inner sep=1pt] {\tiny $1$}
          }
          child {
            node {$3$}
            edge from parent node[fill=white, inner sep=1pt] {\tiny $2$}
          }
          child {
            node {$\cdots$}
          }
      };
  \end{tikzpicture}
\]
\hrulefill
\caption{A computation and a corresponding tree.}
\label{fig:tree}  
\end{figure}

\noindent
In the presence of recursion, some of the leaves of the
tree may also be labelled by $\bot$ to indicate a divergent computation that
does not call any operations. A divergent computation that repeatedly calls
operations is represented by a non-well-founded tree. Denotational semantics
is further discussed in Section~\ref{sec:denotational-semantics}.

\end{description}
