\documentclass{amsart}

\usepackage[T1]{fontenc}
\usepackage[utf8]{inputenc}
\usepackage[hidelinks]{hyperref}
\usepackage{amsmath,amssymb,amsthm}
\usepackage{times}
\usepackage{xypic}

\newcommand{\NN}{\mathbb{N}} % natural numbers
\newcommand{\RR}{\mathbb{R}} % real numbers
\newcommand{\ZZ}{\mathbb{Z}} % integers

\newcommand{\theory}[1]{\mathsf{#1}} % A named theory
\newcommand{\signature}[1]{\Sigma_{\theory{#1}}} % The signature of a theory
\newcommand{\equations}[1]{\mathcal{E}_{\theory{#1}}} % Equations of a theory

\newcommand{\Mod}[1]{\text{\textbf{Mod}}(\theory{#1})} % Models of a theory in Set
\newcommand{\ModC}[2]{\text{\textbf{Mod}}_{\category{#1}}(\theory{#2})} % Models of a theory in a category

\newcommand{\ComodC}[2]{\text{\textbf{CoMod}}_{\category{#1}}(\theory{#2})} % Comodels of a theory in a category

\newcommand{\category}[1]{\text{\textbf{#1}}} % A named category
\newcommand{\Set}{\category{Set}} % The category of sets
\newcommand{\opcat}[1]{\ensuremath{#1^{\text{\textbf{op}}}}}

\newcommand{\carrier}[1]{|#1|} % carrier of a model
\newcommand{\Free}[2]{F_{\theory{#1}}(#2)} % free model
\newcommand{\FreeFun}[1]{F_{\theory{#1}}} % free model functor

\newcommand{\all}[1]{\forall #1 \,.\,} % universal quantifier
\newcommand{\some}[1]{\exists #1 \,.\,} % existential quantifier
\newcommand{\set}[1]{\{#1\}} % set description
\newcommand{\such}{\mid}
\newcommand{\from}{\leftarrow}

\newcommand{\lam}[1]{\lambda #1 \,.\,}

\newcommand{\family}[2]{\{#1\}_{#2}} % a family

\newcommand{\finpow}[1]{\mathcal{P}_{{<}\omega}(#1)} % finite powerset

\newcommand{\Tree}[2]{\mathsf{Tree}_{#1}(#2)} % trees over a signature
\newcommand{\leaf}[1]{\mathsf{leaf}(#1)} % the embedding of generators into trees

\newcommand{\op}[1]{\mathsf{op}_{#1}} % an operation symbol
\newcommand{\arity}[1]{\mathsf{ar}_{#1}} % arity of a symbol
\newcommand{\opdecl}[3]{#1 : #2 \leadsto #3} % operation declaration

\newcommand{\one}{\mathsf{1}} % the terminal object
\newcommand{\unit}{{\text{\small$()$}}} % the sole element of the singleton

\newcommand{\lift}[1]{#1^\dagger} % monad lift

\newcommand{\Cinfty}{\mathcal{C}^\infty}

\newcommand{\sem}[1]{[\![#1]\!]} % semantic bracket

\newcommand{\bool}{\mathsf{bool}} % booleans
\newcommand{\true}{\mathsf{true}}
\newcommand{\false}{\mathsf{false}}
\newcommand{\cond}[3]{\mathsf{if}\;#1\;\mathsf{then}\;#2\;\mathsf{else}\;#3}

\newcommand{\hto}{\Rightarrow} % Handler arrow

\newcommand{\defeq}{\mathbin{{:}{=}}} % definitional equality

%%% Macros for the programming language

% General syntactic constructs
\newcommand{\kode}[1]{\mathsf{#1}}
\newcommand{\conf}[2]{\langle #1, #2 \rangle}

% Core syntax
\newcommand{\seq}[2]{\kode{do}\; #1 \leftarrow #2 \;\kode{in}\;}
\newcommand{\conditional}[3]{\kode{if}\; #1 \;\kode{then}\; #2 \;\kode{else}\; #3}
\newcommand{\fun}[1]{\kode{fun}\; #1 \mapsto}
\newcommand{\handler}{\kode{handler}\;}
\newcommand{\xopgen}[2]{\overline{#1}(#2)}
\newcommand{\opgen}[2]{\xopgen{\kode{#1}}{#2}}
\newcommand{\opcall}[3]{\kode{#1}(#2; #3)}
\newcommand{\opclause}[3]{#1(#2; #3) \mapsto}
\newcommand{\return}[1]{\kode{return}\;#1}
\newcommand{\retclause}[1]{\return{#1} \mapsto}
\newcommand{\withhandle}[2]{\kode{with}\; #1\; \kode{handle}\; #2}



{\theoremstyle{definition}
\newtheorem{definition}{Definition}[section]
\newtheorem{example}[definition]{Example}
}

\begin{document}

\title{What is algebraic about algebraic effects and handlers?}

\author{Andrej Bauer}
\address{Andrej Bauer\\
Faculty of mathematics and Physics\\
University of Ljubljana\\
Jadranska 19\\
1000 Ljubljana\\
Slovenia}
\email{Andrej.Bauer@andrej.com}
\thanks{This material is based upon work supported by the Air Force Office of
  Scientific Research under award number FA9550-17-1-0326.}


\maketitle

This note recapitulates and expands the contents of a tutorial on mathematical
background of algebraic effects and handlers which I gave at the Dagstuhl
seminar \emph{``Algebraic effect handlers go
  mainstream''}~\cite{chandrasekaran18:_algeb} You might find it relevant if you
already are familiar with algebraic effects and handlers as programming concepts
and would like to know what they have to do with algebra.

Our goal is to draw an uninterrupted line of thought between algebra and
computational effects. We begin on the mathematical side of things, by reviewing
the classic notions of universal algebra: signatures, algebraic theories, and
their models. We then generalize and adapt the theory so that it applies to
computational effects. In the last step we replace traditional mathematical
notation with one that is closer to programming languages.


\section{Algebraic theories}
\label{sec:algebraic-theories}


In algebra we study mathematical structures that are equipped with operations satisfying
equational laws. For example, a group is a structure $(G, \mathsf{u}, {\cdot}, {}^{-1})$,
where $\mathsf{u}$~is a constant, $\cdot$~is a binary operation, and ${}^{-1}$~is a unary
operation, satisfying the familiar group identities:
%
\begin{gather*}
  (x \cdot y) \cdot z = x \cdot (y \cdot z),\\
  \mathsf{u} \cdot x = x = x \cdot \mathsf{u},\\
  x \cdot x^{-1} = \mathsf{u} = x^{-1} \cdot x.
\end{gather*}
%
There are alternative axiomatizations, for instance: a group is a monoid
$(G, \mathsf{u}, {\cdot})$ in which every element is invertible, i.e.,
$\all{x \in G} \some{y \in G} x \cdot y = \mathsf{u} = y \cdot x$. However, a
formulation all of whose axioms are equations is preferred, because its simple
logical form grants its models good structural properties.

It is important to distinguish the theory of an algebraic structure from the
algebraic structures it describes. In this section we shall study the
descriptions, which are known as \emph{algebraic} or \emph{equational theories}.

\subsection{Signatures, terms and equations}
\label{sec:signatures-equations}

A \emph{signature $\Sigma$} is a collection of \emph{operation symbols} with
\emph{arities} $\family{(\op{i}, \arity{i})}{i}$. The operation symbols
$\op{i}$ may be any anything, but are usually thought of as syntactic entities,
while arities $\arity{i}$ are non-negative integers. An operation symbol whose
arity is~$0$ is called a \emph{constant} or a \emph{nullary} symbol. Operation
symbols with arities $1$, $2$ and $3$ are referred to as \emph{unary},
\emph{binary}, and \emph{ternary}, respectively.

A (possibly empty) list of distinct variables $x_1, \ldots, x_k$ is called a
\emph{context}. The \emph{$\Sigma$-terms in context $x_1, \ldots, x_k$} are
built inductively using the following rules:
%
\begin{enumerate}
\item each variable $x_i$ is a $\Sigma$-term in context $x_1, \ldots, x_k$,
\item if $t_1, \ldots, t_{\arity{i}}$ are $\Sigma$-terms in context $x_1, \ldots, x_k$ then
  $\op{i}(t_1, \ldots, t_{\arity{i}})$ is a $\Sigma$-term in context $x_1, \ldots, x_k$.
\end{enumerate}
%
We write
%
\begin{equation*}
  x_1, \ldots, x_k \mid t
\end{equation*}
%
to indicate that $t$ is a $\Sigma$-term in the given context. A \emph{closed
  $\Sigma$-term} is a $\Sigma$-term in the empty context. No variables occur
in a closed term.

A \emph{$\Sigma$-equation} is a pair of $\Sigma$-terms $\ell$ and $r$ in context
$x_1, \ldots, x_k$. We write
%
\begin{equation*}
  x_1, \ldots, x_k \mid \ell = r
\end{equation*}
%
to indicate an equation in a context. We shall often elide the context and write simply
$\ell = r$, but it should be understood that there is an ambient context which contains at
least all the variables mentioned by $\ell$ and $r$.

A $\Sigma$-equation really is just a list of variables and a pair of terms, and
\emph{not} a logical statement. The context variables are \emph{not} universally
quantified, and we are not talking about first-order logic. Of course, a
$\Sigma$-equation is suggestively written as an equation because we do
eventually want to \emph{interpret} it as an assertion of equality, but until
such time (and even afterwards) it is better to think of contexts, terms, and
equations as ordinary mathematical objects, devoid of any imagined or special
meta-mathematical status. This remark will hopefully become clearer in
Section~\ref{sec:algebr-theor-with}.

When no confusion can arise we drop the prefix ``$\Sigma$-'' and simply speak about
terms and equations instead of $\Sigma$-terms and $\Sigma$-equations.


\begin{example}
  \label{ex:monoid-signature}
  %
  The signature for the theory of a monoid has a nullary symbol $\mathsf{u}$ and a binary
  symbol $\mathsf{m}$. There are infinitely many expressions in context $x, y$, such as
  %
  \begin{align*}
    \mathsf{u}(),\quad
    x,\quad
    y,\quad
    \mathsf{m}(\mathsf{u}(), \mathsf{u}()),\quad
    \mathsf{m}(\mathsf{u}(), x),\quad
    \mathsf{m}(y, \mathsf{u}()),\quad
    \mathsf{m}(x, x),\quad
    \mathsf{m}(y, x),
    \ldots
  \end{align*}
  %
  An equation in context $x, y$ is
  %
  \begin{equation*}
    x, y \mid \mathsf{m}(y, x) = \mathsf{m}(\mathsf{m}(\mathsf{u}(), x), y).
  \end{equation*}
  %
  It is customary to write a nullary symbol $\mathsf{u}()$ simply as $\mathsf{u}$, and to
  use infix an infix operator~$\cdot$ in place of~$\mathsf{m}$. With such notation the
  above equation would be written as
  %
  \begin{equation*}
    x, y \mid y \cdot x = (\mathsf{u} \cdot x) \cdot y.
  \end{equation*}
  %
  One might even omit $\cdot$ and the context, in which case the equation is
  written simply as $y \, x = (\mathsf{u} \, x) \, y$. If we agree that $\cdot$
  associates to the left then $(\mathsf{u} \, x) \, y$ may be written as
  $\mathsf{u} \, x \, y$, and we are left with $y \, x = \mathsf{u} \, x \, y$,
  which is what your algebra professor might write down. Note that we are
  \emph{not} discussing validity of equations but only ways of displaying them.
  It is irrelevant whether the above equation is valid in monoids.
\end{example}


\subsection{Algebraic theories}
\label{sec:algebraic-theories-1}

An \emph{algebraic theory $\theory{T} = (\signature{T}, \equations{T})$} is given by a
signature~$\signature{T}$ and a collection $\equations{T}$ of $\signature{T}$-equations.
%
We impose no restrictions on the number of operation symbols or equations, but at least in
classical treatments of the subject certain complications are avoided by insisting that
arities be non-negative integers.

\begin{example}
  \label{ex:theory-group}
  %
  The theory~$\theory{Group}$ of a group is algebraic. In order to follow closely the
  definitions we eschew the traditional notation $\cdot$ and ${}^{-1}$, and explicitly
  display the contexts. We abide by such formalistic requirements once to demonstrate
  them, but shall take notational liberties subsequently.
  %
  The signature $\signature{Group}$ is given by operation symbols $\mathsf{u}$,
  $\mathsf{m}$, and $\mathsf{i}$ whose arities are $0$, $2$, and $1$, respectively. The
  equations $\equations{Group}$ are:
  %
  \begin{align*}
    x, y, z &\mid \mathsf{m}(\mathsf{m}(x, y), z) = \mathsf{m}(x, \mathsf{m}(y, z)),\\
    x &\mid \mathsf{m}(\mathsf{u}(), x) = x \\
    x &\mid \mathsf{m}(x, \mathsf{u}()) = x,\\
    x &\mid \mathsf{m}(x, \mathsf{i}(x)) = \mathsf{u}()\\
    x &\mid \mathsf{m}(\mathsf{i}(x), x) = \mathsf{u}().
  \end{align*}
  %
\end{example}

\begin{example}
  \label{ex:semi-lattice}
  %
  The theory $\theory{Semilattice}$ of a semilattice is algebraic. It is given by a
  nullary symbol $\bot$ and a binary symbol $\vee$, satisfying the equations
  %
  \begin{align*}
    x \vee (y \vee z) &= (x \vee y) \vee z,\\
    x \vee y &= y \vee x,\\
    x \vee x &= x,\\
    x \vee \bot &= x.
  \end{align*}
  %
  It should be clear that the first equation has context $x, y, z$, the second one
  in~$x, y$, and the last two in~$x$.
\end{example}

\begin{example}
  \label{ex:field}
  %
  The theory of a field, as usually given, is not algebraic because the inverse~$0^{-1}$
  is undefined, whereas the operations of an algebraic theory are always taken to be
  total. However, a proof is required to show that there is no equivalent algebraic theory.
\end{example}

\begin{example}
  \label{ex:pointed-set}
  %
  The theory $\theory{Set_\bullet}$ of a \emph{pointed set} has a constant $\bullet$ and
  no equations.
\end{example}

\begin{example}
  \label{ex:theory-empty}
  %
  The \emph{empty theory $\theory{Empty}$} has no operation symbols and no equations.
\end{example}

\begin{example}
  \label{ex:theory-singleton}
  %
  The theory of a \emph{singleton $\theory{Singleton}$} has a constant $\star$ and the
  equation $x = y$.
\end{example}

\begin{example}
  \label{ex:lattice}
  %
  A bounded lattice is a partial order with finite infima and suprema. Such a formulation
  is not algebraic because the infimum and supremum operators do not have fixed arities,
  but we can reformulate it in terms of nullary and binary operations. Thus, the theory
  $\theory{Lattice}$ of a bounded lattice has constants $\bot$ and $\top$, and two binary
  operation symbols $\vee$ and $\wedge$, satisfying the equations:
  %
  \begin{align*}
    x \vee (y \vee z) &= (x \vee y) \vee z,   &      x \wedge (y \wedge z) &= (x \wedge y) \wedge z,\\
    x \vee y &= y \vee x,                     &      x \wedge y &= y \wedge x,\\
    x \vee x &= x,                            &      x \wedge x &= x,\\
    x \vee \bot &= x,                         &      x \wedge \top &= x.
  \end{align*}
  %
  Notice that the theory of a bounded lattice is simply the juxtaposition of two copies of
  the theory of a semi-lattice from Example~\ref{ex:semi-lattice}. The partial order is
  recovered because $x \leq y$ is equivalent to $x \vee y = y$ and to $x \wedge y = x$.
\end{example}

\begin{example}
  \label{ex:finitely-generated-group}
  %
  A \emph{finitely generated group} is a group which contains a finite collection of
  elements, called the \emph{generators}, such that every element of the group is obtained
  by multiplications and inverses of the generators. It is not clear how to express this
  condition using only equations, but a proof is required to show that there is no
  equivalent algebraic theory.
\end{example}

\begin{example}
  \label{ex:Cinfty-theory}
  %
  An example of an algebraic theory with many operations and equations is the theory of a
  $\Cinfty$-ring. Let $\Cinfty(\RR^n, \RR^m)$ be the set of all smooth maps from $\RR^n$
  to $\RR^m$. The signature for the theory of a $\Cinfty$-ring contains an $n$-ary
  operation symbol $\op{f}$ for each $f \in \Cinfty(\RR^n, \RR)$. For all
  $f \in \Cinfty(\RR^n, \RR)$, $h \in \Cinfty(\RR^m, \RR)$, and
  $g_1, \ldots, g_n \in \Cinfty(\RR^m, \RR)$ such that
  %
  \begin{equation*}
    f \circ (g_1, \ldots, g_n) = h,
  \end{equation*}
  %
  the theory has the equation
  %
  \begin{equation*}
    x_1, \ldots, x_m \mid
    \op{f} (\op{g_1}(x_1, \ldots, x_m), \ldots, \op{g_n}(x_1, \ldots, x_m)) =
    \op{h}(x_1, \ldots, x_m).
  \end{equation*}
  %
  The theory contains the theory of a commutative unital ring as a subtheory. Indeed,
  the ring operations on~$\RR$ are smooth maps, and so they appear as $\op{+}$,
  $\op{\times}$, $\op{-}$ in the signature, and so do constants $\op{0}$ and $\op{1}$,
  because all maps $\RR^0 \to \RR$ are smooth. The commutative ring equations are present
  as well because the real numbers form a commutative ring.
\end{example}


\subsection{Interpretations of signatures}
\label{sec:interp-of-sign}

Let a signature $\Sigma$ be given. An \emph{interpretation~$I$} of $\Sigma$ is given by
the following data:
%
\begin{enumerate}
\item a set $\carrier{I}$, called the \emph{carrier},
\item for each operation symbol $\op{i}$ a map
  %
  \begin{equation*}
    \sem{\op{i}}_I : \underbrace{\carrier{I} \times \cdots \times \carrier{I}}_{\arity{i}} \to \carrier{I}.
  \end{equation*}
\end{enumerate}
%
The double bracket $\sem{{\ }}_I$ is called the \emph{semantic bracket} and is typically
used when syntactic entities (operation symbols, terms, equations) are mapped by~$I$ to
their mathematical counterparts. When no confusion can arise, we omit the subscript~$I$
and write just~$\sem{{\ }}$.

We abbreviate an $n$-ary product $\carrier{I} \times \cdots \times \carrier{I}$ as $\carrier{I}^n$. A
nullary product $\carrier{I}^0$ contains a single element, namely the empty tuple
$\unit$, so it makes sense to write $\carrier{I}^0 = \one = \set{\unit}$. Thus a nullary
operation symbol is interpreted by an map $\one \to \carrier{I}$, and such maps are in
bijective correspondence with the elements of~$\carrier{I}$, which would be the constants.

An interpretation~$I$ may be extended to $\Sigma$-terms. A $\Sigma$-term in context
%
\begin{equation*}
  x_0, \ldots, x_{k-1} \mid t
\end{equation*}
%
is interpreted by a map
%
\begin{equation*}
  \sem{x_0, \ldots, x_{k-1} \mid t}_I : \carrier{I}^k \to \carrier{I},
\end{equation*}
%
as follows:
%
\begin{enumerate}
\item the variable $x_i$ is interpreted as the $i$-th projection,
  %
  \begin{equation*}
    \sem{x_0, \ldots, x_{k-1} \mid  x_i}_I = \pi_i : \carrier{I}^k \to \carrier{I},
  \end{equation*}
\item a compound term in context
  %
  \begin{equation*}
    x_0, \ldots, x_{k-1} \mid \op{i}(t_1, \ldots, t_{\arity{i}})
  \end{equation*}
  %
  is interpreted as the composition of maps
  %
  \begin{equation*}
    \xymatrix@+6em{
      {\carrier{I}^k} \ar[r]^{(\sem{t_1}_I, \ldots, \sem{t_{\arity{i}}}_I)}
      &
      {\carrier{I}^{\arity{i}}} \ar[r]^{\sem{\op{i}}_I}
      &
      {\carrier{I}}
    }
  \end{equation*}
  %
  where we elided the contexts $x_0, \ldots, x_{k-1}$ for the sake of brevity.
\end{enumerate}

\begin{example}
  One interpretation of the signature from Example~\ref{ex:monoid-signature} is given by
  the carrier set $\RR$ and the interpretations of operation symbols
  %
  \begin{align*}
    \sem{\mathsf{u}}() &= 1 + \sqrt{5}, \\
    \sem{\mathsf{m}}(a, b) &= a^2 + b^3.
  \end{align*}
  %
  The term in context $x, y \mid \mathsf{m}(\mathsf{u}, \mathsf{m}(x, x))$ is interpreted
  as the map $\RR \to \RR$, given by the rule
  %
  \begin{equation*}
    (a, b) \mapsto (a+1)^3 a^6 + 2 (3 + \sqrt{5}).
  \end{equation*}
  %
  The same term in a context $y, x, z$ is interpreted as the map $\RR \times \RR \to \RR$,
  given by the rule
  %
  \begin{align*}
    (a, b, c) &\mapsto (b+1)^3 b^6 + 2 (3 + \sqrt{5}).
  \end{align*}
  %
  These are not the same map, as they do not even have the same domains! It is irrelevant
  whether this interpretation satisfies the monoid laws because we have not considered any
  equations yet.
\end{example}

The previous examples shows why contexts should not be ignored. In mathematical practice
contexts are often relegated to guesswork for the reader, or are handled implicitly. For
example, in real algebraic geometry the solution set of the equation $x^2 + y^2 = 1$ is
either a unit circle in the plane or an infinitely extending cylinder of unit radius in
the space, depending on whether the context might be $x, y$ or $x, y, z$. Which context is
meant is indicated one way or another by the author of the mathematical text.

\subsection{Models of algebraic theories}
\label{sec:models-algebr-theor}

A \emph{model~$M$} of an algebraic theory~$\theory{T}$ is an interpretation of the signature
$\signature{T}$ which validates all the equations $\equations{T}$. That is, for every
equation
%
\begin{equation*}
  x_1, \ldots, x_k \mid \ell = r
\end{equation*}
%
in~$\equations{T}$, the maps
%
\begin{equation*}
  \sem{x_1, \ldots, x_k \mid \ell}_M : \carrier{M}^k \to \carrier{M}
  \qquad\text{and}\qquad
  \sem{x_1, \ldots, x_k \mid r}_M : \carrier{M}^k \to \carrier{M}
\end{equation*}
%
are equal. We refer to a model of~$\theory{T}$ as a \emph{$\theory{T}$-model} or
a \emph{$\theory{T}$-algebra}.

\begin{example}
  A model $G$ of $\theory{Group}$, cf.\ Example~\ref{ex:theory-group}, is given by a
  carrier set $\carrier{G}$ and maps
  %
  \begin{equation*}
    u : \one \to \carrier{G},\qquad
    m : \carrier{G} \times \carrier{G} \to \carrier{G},\qquad
    i : \carrier{G} \to \carrier{G},
  \end{equation*}
  %
  interpreting the operation symbols $\mathsf{u}$, $\mathsf{m}$, and $\mathsf{i}$,
  respectively, such that the equations~$\equations{Group}$. This amounts precisely to
  $(\carrier{G}, u, m, i)$ being a group, except that the unit $u$ is viewed as a map
  $\one \to \carrier{G}$ instead of an element of~$\carrier{G}$.
\end{example}

\begin{example}
  Every algebraic theory has the \emph{trivial model}, whose carrier is the
  singleton~$\one$, and whose operations are interpreted by the unique maps
  $\one^k \to \one$. All equations are satisfied because any two maps $\one^k \to \one$
  are equal.
\end{example}

The previous example explains why one should \emph{not} require $0 \neq 1$ in a ring, as
that prevents the theory of a ring from being algebraic.

\begin{example}
  The empty set is a model of a theory~$\theory{T}$ if, and only if, every operation symbol
  of $\theory{T}$ has non-zero arity.
\end{example}

\begin{example}
  A model of the theory~$\theory{Set_\bullet}$ of a pointed set, cf.\
  Example~\ref{ex:pointed-set}, is a set $S$ together with an element $s \in S$ which
  interprets the constant~$\bullet$.
\end{example}

\begin{example}
  A model of the theory~$\theory{Empty}$, cf.\ Example~\ref{ex:theory-empty}, is the same
  thing as a set.
\end{example}

\begin{example}
  A model of the theory~$\theory{Singleton}$, cf.\ Example~\ref{ex:theory-singleton}, is
  any set with precisely one element.
\end{example}

Suppose $L$ and $M$ are models of a theory~$\theory{T}$. Then we may form the
\emph{product of models} $L \times M$ by taking the cartesian product as the carrier,
%
\begin{equation*}
  \carrier{L \times M} = \carrier{L} \times \carrier{M},
\end{equation*}
%
and pointwise operations,
%
\begin{equation*}
  \sem{\op{i}}_{M \times L}(a, b) = (\sem{\op{i}}_M(a), \sem{\op{i}}_L(b)).
\end{equation*}
%
The equations $\equations{T}$ are valid in $L \times M$ because they are valid on each
coordinate separately. This construction can be extended to a product of any number of
models, including an infinite one.

\begin{example}
  We may now prove that the theory of a field from Example~\ref{ex:field} is not
  equivalent to an algebraic theory. There are fields of size 2 and 3, namely $\ZZ_2$ and
  $\ZZ_3$. If there were an algebraic theory of a field, then $\ZZ_2 \times \ZZ_3$ would
  be a field too, but it is not, and in fact there is no field of size~6.
\end{example}

\begin{example}
  Similarly, the theory of a finitely generated group from
  Example~\ref{ex:finitely-generated-group} cannot be formulated as an algebraic theory,
  because an infinite product of non-trivial finitely generated groups is not finitely
  generated.
\end{example}

\begin{example}
  Let us give a model of the theory of a $\Cinfty$-ring from
  Example~\ref{ex:Cinfty-theory}. Pick a smooth manifold~$M$, and let the carrier be the
  set $\Cinfty(M, \RR)$ of all smooth scalar fields on~$M$. Given
  $f \in \Cinfty(\RR^n, \RR)$, interpret the operation $\op{f}$ as composition with~$f$,
  %
  \begin{align*}
    \sem{\op{f}} &: \Cinfty(M, \RR)^n \to \Cinfty(M, \RR) \\
    \sem{\op{f}} &: (u_1, \ldots, u_n) \mapsto f \circ (u_1, \ldots, u_n).
  \end{align*}
  %
  We leave it as an exercise to verify that all equations are validated by this
  interpretation.
\end{example}

\subsection{Homomorphisms and the category of models}
\label{sec:homom-categ-models}

Suppose $L$ and $M$ are models of a theory~$\theory{T}$. A
\emph{$\theory{T}$-homomorphism} from $L$ to $M$ is a map $\phi : \carrier{L} \to \carrier{M}$ between the
carriers which commutes with operations: for every operation symbol $\op{i}$
of~$\theory{T}$, we have
%
\begin{equation*}
  \phi \circ \sem{\op{i}}_L = \sem{\op{i}}_M \circ \underbrace{(\phi, \ldots, \phi)}_{\arity{i}}.
\end{equation*}

\begin{example}
  A homomorphism between groups $G$ and $H$ is a map $\phi : \carrier{G} \to \carrier{H}$ between the
  carriers such that, for all $a, b \in \carrier{G}$,
  %
  \begin{align*}
    \phi(\sem{\mathsf{u}()}_G) &= \sem{\mathsf{u}()}_H,\\
    \phi(\sem{\mathsf{m}}_G(a,b)) &= \sem{\mathsf{m}}_H(\phi(a), \phi(b)),\\
    \phi(\sem{\mathsf{i}}_G(a)) &= \sem{\mathsf{i}}_H(\phi(a)).
  \end{align*}
  %
  This is a convoluted way of saying that the unit maps to the unit, and that $\phi$
  commutes with the group operation and the inverses. Textbooks usually require only that
  a group homomorphism commute with the group operation, which then implies that it also
  preserves the unit and commutes with the inverse.
\end{example}

We may organize the models of an algebraic theory~$\theory{T}$ into a category $\Mod{T}$
whose objects are the models of the theory, and whose morphisms are homomorphisms of the
theory.

\begin{example}
  The category of models of theory $\theory{Group}$, cf.\ Example~\ref{ex:theory-group},
  is the usual category of groups and group homomorphisms.
\end{example}

\begin{example}
  The category of models of the theory $\theory{Set_\bullet}$, cf.\
  Example~\ref{ex:pointed-set}, has as its objects the pointed sets, which are pairs
  $(S, s)$ with $S$ a set and $s \in S$ its \emph{point}, and as homomorphisms
  the point-preserving functions between sets.
\end{example}

\begin{example}
  The category of models of the empty theory $\theory{Empty}$, cf.\
  Example~\ref{ex:theory-empty}, is just the category $\category{Set}$ of sets and
  functions.
\end{example}

\begin{example}
  The category of models of the theory of a singleton $\theory{Singleton}$, cf.\
  Example~\ref{ex:theory-singleton}, is the category whose objects are all the singleton
  sets. There is precisely one morphisms between any two of them. This category is
  equivalent to the trivial category which has just one object and one morphism.
\end{example}

\subsection{Models in a category}
\label{sec:models-category}

So far we have taken the models of an algebraic theory to be sets. More generally, we may
consider models in any category $\category{C}$ with finite products. Indeed, the
definitions of an interpretation and a model from Sections~\ref{sec:interp-of-sign}
and~\ref{sec:models-algebr-theor} may be directly transcribed so that they apply
to~$\category{C}$. An interpretation $I$ in $\category{C}$ is given by
%
\begin{enumerate}
\item an object $\carrier{I}$ in $\category{C}$, called the \emph{carrier},
\item for each operation symbol $\op{i}$ a morphism in $\category{C}$
  %
  \begin{equation*}
    \sem{\op{i}}_I : \underbrace{\carrier{I} \times \cdots \times \carrier{I}}_{\arity{i}} \to \carrier{I}.
  \end{equation*}
\end{enumerate}
%
Once again, we abbreviate the $k$-fold product of $\carrier{I}$ as $\carrier{I}^k$. Notice that a nullary
symbol is interpreted as a morphism $\carrier{I}^0 \to \carrier{I}$, which is a morphisms from the
terminal object $\one \to \carrier{I}$ in~$\category{C}$.

An interpretation~$I$ is extended to $\Sigma$-terms in contexts as follows:
%
\begin{enumerate}
\item the variable $x_1, \ldots, x_k \mid x_i$ is interpreted as the $i$-th projection,
  %
  \begin{equation*}
    \sem{x_0, \ldots, x_{k-1} \mid  x_i}_I = \pi_i : \carrier{I}^k \to \carrier{I},
  \end{equation*}
\item a compound term in context
  %
  \begin{equation*}
    x_0, \ldots, x_{k-1} \mid \op{i}(t_1, \ldots, t_{\arity{i}})
  \end{equation*}
  %
  is interpreted as the composition of morphisms
  %
  \begin{equation*}
    \xymatrix@+6em{
      {\carrier{I}^k} \ar[r]^{(\sem{t_0}_I, \ldots, \sem{t_{\arity{i}}}_I)}
      &
      {\carrier{I}^{\arity{i}}} \ar[r]^{\sem{\op{i}}_I}
      &
      {\carrier{I}}
    }
  \end{equation*}
\end{enumerate}
%
A model of an algebraic theory~$\theory{T}$ in~$\category{C}$ is an interpretation~$M$ of
its signature $\signature{T}$ which validates all the equations. That is, for every
equation
%
\begin{equation*}
  x_1, \ldots, x_k \mid \ell = r
\end{equation*}
%
in $\equations{T}$, the morphisms
%
\begin{equation*}
  \sem{x_1, \ldots, x_k \mid \ell}_M : \carrier{M}^k \to \carrier{M}
  \qquad\text{and}\qquad
  \sem{x_1, \ldots, x_k \mid r}_M : \carrier{M}^k \to \carrier{M}
\end{equation*}
%
are equal.

The definition of a homomorphism carries over to the general setting as well. A
\emph{$\theory{T}$-homomorphism} between $\theory{T}$-models $L$ and $M$ in a category
$\category{C}$ is a morphism $\phi : \carrier{L} \to \carrier{M}$ in~$\category{C}$ such that, for every
operation symbol~$\op{i}$ in~$\theory{T}$, $\phi$ commutes with the interpretation of
$\op{i}$,
%
\begin{equation*}
  \phi \circ \sem{\op{i}}_L = \sem{\op{i}}_M \circ \underbrace{(\phi, \ldots, \phi)}_{\arity{i}}.
\end{equation*}
%
The $\theory{T}$-models and $\theory{T}$-homomorphisms in a category $\category{C}$ form a
category $\ModC{C}{T}$.

\begin{example}
  A model of the theory of a group $\theory{Group}$ in the category $\category{Top}$ of
  topological spaces and continuous maps is a topological group.
\end{example}

\begin{example}
  What is a model of the theory of a group in the category of groups $\category{Grp}$? Its
  carrier is a group $(G, u, m, i)$ together with group homomorphisms
  $\upsilon : \one \to G$, $\mu : G \times G \to G$, and $\iota : G \to G$ which satisfy
  the group laws. Because $\upsilon$ is a group homomorphism, it maps the unit of the
  trivial group~$\one$ to $u$, so the units $u$ and $\upsilon$ agree. The operations $m$
  and $\mu$ agree too, because
  %
  \begin{equation*}
    \mu(x, y) =
    \mu(m(x, u), m(u, y)) =
    m(\mu(x, u), \mu(u, y)) =
    m(x, y),
  \end{equation*}
  %
  where in the middle step we used the fact that $\mu$ is a group homomorphism. It is now
  clear that the inverses $i$ and $\iota$ agree as well. Furthermore, taking into account
  that $m$ and $\mu$ agree, we also obtain
  %
  \begin{equation*}
    m(x, y) =
    m(m(u, x), m(y, u)) =
    m(m(u, y), m(x, u)) =
    m(y, x).
  \end{equation*}
  %
  The conclusion is that a group in the category of groups is an abelian group. The
  category $\ModC{Grp}{Group}$ is therefore equivalent to the category of abelian groups.
\end{example}

\begin{example}
  A model of the theory of a pointed set, cf.\ Example~\ref{ex:pointed-set}, in the
  category of groups $\category{Grp}$ is a group $(G, u, m, i)$ together with a
  homomorphism $\one \to G$ from the trivial group~$\one$ to $G$. However, there is
  precisely one such homomorphism which therefore need not be mentioned at all. Thus a
  pointed set in groups amounts to a group.
\end{example}


\subsection{Free models}
\label{sec:free-models}

Of special interest are the free models of an algebraic theory. Given an
algebraic theory~$\theory{T}$ and a set $X$, the \emph{free $\theory{T}$-model},
also called the \emph{free $\theory{T}$-algebra}, generated by~$X$ is a model
$M$ together with a map $\eta : X \to \carrier{M}$ such that, for every
$\theory{T}$-model $L$ and every map $f : X \to \carrier{L}$ there is a unique
$\theory{T}$-homomorphism $\overline{f} : M \to L$ for which the following
diagram commutes:
%
\begin{equation*}
  \xymatrix{
    {X}
    \ar[r]^{\eta}
    \ar[rd]_{f}
    &
    {\carrier{M}}
    \ar[d]^{\overline{f}}
    \\
    &
     \carrier{L}
  }
\end{equation*}
%
The definition is a bit of a mouthful, but it can be understood as follows: the
free $\theory{T}$-model generated by $X$ is the ``most economical'' way of
making a $\theory{T}$-model out of the set~$X$.

\begin{example}
  The free group generated by the empty set is the trivial group~$\one$ with
  just one element. The map $\eta : \emptyset \to \one$ is the unique one, and
  given any (unique) map $f : \emptyset \to \carrier{G}$ to a carrier of another
  group~$G$, there is a unique group homomorphism $\overline{f} : \one \to G$.
  The relevant triangle commutes automatically because it originates
  at~$\emptyset$.
\end{example}

\begin{example}
  The free group generated by the singleton set $\one$ is the group of integers
  $(\ZZ, 0, {+}, {-})$. The map $\eta : \set{\star} \to \ZZ$ takes the generator
  $\unit$ to $0$. As an exercise you should verify that the integers have the
  required universal property.
\end{example}

\begin{example}
  Let $\finpow{X}$ be the set of all finite subsets of a set~$X$. We show that
  $(\finpow{X}, \emptyset, {\cup})$ is the free semilattice generated by~$X$, cf.\
  Example~\ref{ex:semi-lattice}. The map $\eta : X \to \finpow{X}$ takes $x \in X$ to the
  singleton set $\eta(x) = \set{x}$. Given any semilattice $(L, \bot, {\vee})$ and a map
  $f : X \to \carrier{L}$, define the homomorphism $\overline{f} : \finpow{X} \to \carrier{L}$ by
  %
  \begin{equation*}
    \overline{f}(\set{x_1, \ldots, x_n}) = f(x_1) \wedge \cdots \wedge f(x_n).
  \end{equation*}
  %
  Clearly, the required diagram commutes because
  %
  \begin{equation*}
    \overline{f}(\eta(x)) = \overline{f}(\set{x}) = f(x).
  \end{equation*}
  %
  If $g : \finpow{X} \to \carrier{L}$ is another homomorphism satisfying $g \circ \eta = f$ then
  %
  \begin{multline*}
    g(\set{x_1, \ldots, x_n})
    = g(\eta(x_1) \cup \cdots \cup \eta(x_n))
    = g(\eta(x_1)) \wedge \cdots \wedge g(\eta(x_n)) \\
    = f(x_1) \wedge \cdots \wedge f(x_n)
    = \overline{f}(\set{x_1, \ldots, x_n}),
  \end{multline*}
  %
  hence $\overline{f}$ is indeed unique.
\end{example}

\begin{example}
  The free model generated by~$X$ of the theory of a pointed set, cf.\
  Example~\ref{ex:pointed-set}, is the disjoint union $X + \one$ whose elements are of the
  form $\iota_0(x)$ for $x \in X$ and $\iota_1(y)$ for $y \in \one$. The point is the
  element $\iota_1()$. The map $\eta : X \to X + \one$ is the canonical
  inclusion~$\iota_0$.
\end{example}

\begin{example}
  The free model generated by~$X$ of the empty theory, cf.\ Example~\ref{ex:theory-empty},
  is~$X$ itself, with $\eta : X \to X$ the identity map.
\end{example}

\begin{example}
  The free model generated by~$X$ of the theory of a singleton, cf.\
  Example~\ref{ex:theory-singleton}, is the singleton set~$\one$, with $\eta : X \to \one$
  the only map it could be. This example shows that~$\eta$ need not be injective.
\end{example}

Every algebraic theory~$\theory{T}$ has free models. Let us sketch its
construction. Given a signature $\Sigma$ and a set~$X$,
define~$\Tree{\Sigma}{X}$ to be the set of well-founded trees built inductively
as follows:
%
\begin{enumerate}
\item for each $x \in X$, there is a tree $\leaf{x} \in \Tree{\Sigma}{X}$,
\item for each operation symbol $\op{i}$ and trees
  $t_1, \ldots, t_{\arity{i}} \in \Tree{\Sigma}{X}$, there is a tree
  $\op{i}(t_1, \ldots, t_n) \in \Tree{\Sigma}{X}$.
\end{enumerate}
%
Notice that the $\Sigma$-terms in context $x_1, \ldots, x_n$ are precisely the trees
in $\Tree{\Sigma}{\set{x_1, \ldots, x_n}}$.

Suppose $x_1, \ldots, x_n \mid t$ is a $\Sigma$-term in context, and we are
given an assignment $\sigma : \set{x_1, \ldots, x_n} \to \Tree{\Sigma}{X}$ of
trees to variables. Then we may build the tree $\sigma(t)$ inductively as
follows:
%
\begin{enumerate}
\item $\sigma(t) = \sigma(x_i)$ if $t = x_i$,
\item $\sigma(t) = \op{i}(\sigma(t_1), \ldots, \sigma(t_n))$ if
  $t = \op{i}(t_1, \ldots, t_n)$.
\end{enumerate}
%
In words, the tree $\sigma(t)$ is obtained by replacing each variable~$x_i$ in~$t$ with
the corresponding tree $\sigma(x_i)$.


Given a theory~$\theory{T}$, let $\approx_\theory{T}$ be the least equivalence relation on
$\Tree{\signature{T}}{X}$ such that:
%
\begin{enumerate}
\item for every equation $x_1, \ldots, x_n \mid \ell = r$ in $\equations{T}$ and for every
  assignment $\sigma : \set{x_1, \ldots, x_n} \to \Tree{\signature{T}}{X}$, we have
  %
  \begin{equation*}
    \sigma(\ell) \approx_{\theory{T}} \sigma(r).
  \end{equation*}
  %
\item $\approx_{\theory{T}}$ is a \emph{$\signature{T}$-congruence}: for every
  operation symbol $\op{i}$ in $\signature{T}$, and for all trees
  $s_1, \ldots, s_{\arity{i}}$ and $t_1, \ldots, t_{\arity{i}}$, if
  %
  \begin{equation*}
    s_1 \approx_{\theory{T}} t_1,
    \quad \ldots \quad,
    s_{\arity{i}} \approx_{\theory{T}} t_{\arity{i}}
  \end{equation*}
  %
  then
  %
  \begin{equation*}
    \op{i}(s_1, \ldots, s_{\arity{i}}) \approx_{\theory{T}}
    \op{i}(t_1, \ldots, t_{\arity{i}}).
  \end{equation*}
\end{enumerate}
%
Define the carrier of the free model $\Free{T}{X}$ to be the quotient set
%
\begin{equation*}
  \carrier{\Free{T}{X}} = \Tree{\signature{T}}{X} / {\approx_{\theory{T}}}.
\end{equation*}
%
Let $[t]$ be the $\approx_{\theory{T}}$-equivalence class of
$t \in \Tree{\signature{T}}{X}$. The interpretation of the operation symbol $\op{i}$ in
  $\Free{T}{X}$ is the map $\sem{\op{i}}_{\Free{T}{X}}$ defined by
%
\begin{equation*}
  \sem{\op{i}}_{\Free{T}{X}}([t_1], \ldots, [t_{\arity{i}}]) =
  [\op{i}(t_1, \ldots, t_{\arity{i}})].
\end{equation*}
%
The map $\eta_X : X \to \Free{T}{X}$ is defined by
%
\begin{equation*}
  \eta_X(x) = [\leaf{x}].
\end{equation*}
%
To see that we successfully defined a $\theory{T}$-model, and that it is freely generated
by~$X$, one has to verify a number of mostly straightforward technical details, which we
omit.

When a theory $\theory{T}$ has no equations the free models generated by~$X$ is just the
set of trees $\Tree{T}{X}$ because the relation $\approx_{\theory{T}}$ is equality.

\subsection{Operations with general arities and parameters}
\label{sec:oper-gener-arit-param}

We have so far followed the classic mathematical presentation of algebraic
theories. To get a better fit with computational effects, we need to generalize
operations in two ways.

\subsubsection{General arities}
\label{sec:general-arities}

We shall require operations that accept an arbitrary, but fixed collection of
arguments. One might expect that the correct way to do so is to allow arities to
be ordinal or cardinal numbers, as these generalize natural numbers, but that
would be a thoroughly non-computational idea. Instead, let us observe that an
$n$-ary cartesian product
%
\begin{equation*}
  \underbrace{X \times \cdots \times X}_{n}
\end{equation*}
%
is isomorphic to the exponential $X^{[n]}$, where
$[n] = \set{0, 1, \ldots, n-1}$. Recall that an exponential $B^A$ is the set of
all functions $A \to B$, and in fact we shall use the notations $B^A$ and
$A \to B$ interchangeably. If we replace $[n]$ by an arbitrary set~$A$, then we
can think of a map
%
\begin{equation*}
  X^A \to X
\end{equation*}
%
as taking $A$-many arguments. We need reasonable notation for writing down an
operation symbol applied to $A$-many arguments, where $A$ is an arbitrary set. One
might be tempted to adapt the tuple notation and write something silly, such as
%
\begin{equation*}
  \op{i}(\cdots t_a \cdots)_{a \in A},
\end{equation*}
%
but as computer scientists we know better than that. Let us use the notation that
is already provided to us by the exponentials, namely the $\lambda$-calculus. To
have $A$-many elements of a set $X$ is to have a map $\kappa : A \to X$, and
thus to apply the operation symbol $\op{i}$ to $A$-many arguments $\kappa$ we
simply write~$\op{i}(\kappa)$.


\begin{example}
  Let us rewrite the group operations in the new notation. The empty set
  $\emptyset$, the singleton $\one$, and the set of boolean values
  %
  \begin{equation*}
    \bool = \set{\false, \true}
  \end{equation*}
  %
  serve as arities. We use the conditional statement
  %
  \begin{equation*}
    \cond{b}{x}{y}
  \end{equation*}
  %
  as a synonym for what is usually written as definition by cases,
  %
  \begin{equation*}
  \begin{cases}
      x & \text{if $b = \true$,}\\
      y & \text{if $b = \false$.}
    \end{cases}
  \end{equation*}
  %
  Now a group is given by a carrier set $G$ together with maps
  %
  \begin{align*}
    \mathsf{u} &: G^\emptyset \to G,\\
    \mathsf{m} &: G^\bool \to G,\\
    \mathsf{i} &: G^\one \to G,
  \end{align*}
  %
  satisfying the usual group laws, which we ought to write down using the
  $\lambda$-notation. The associativity law is written like this:
  %
  \begin{multline*}
    \mathsf{m}(\lam{b} \cond{b}{\mathsf{m}(\lam{c}\cond{c}{x}{y})}{z}) = \\
    \mathsf{m}(\lam{b} \cond{b}{x}{\mathsf{m}(\lam{c} \cond{c}{y}{z})}).
  \end{multline*}
  %
  Here is the right inverse law, where $\mathsf{O}_X : \emptyset \to X$ is
  the unique map from $\emptyset$ to~$X$:
  %
  \begin{equation*}
    \mathsf{m}(\lam{b} \cond{b}{x}{\mathsf{i}(\lam{\_}{x})}) =
    \mathsf{u}(\mathsf{O}_G).
  \end{equation*}
  %
  The symbol $\_$ indicates that the argument of the $\lambda$-abstraction is
  ignored, i.e., that the function defined by the abstraction is constant. One
  more example might help: $x$ squared may be written as
  $\mathsf{m}(\lam{b} \cond{b}{x}{x})$ as well as $\mathsf{m}(\lam{\_} x)$.
\end{example}

Such notation is not appropriate for performing algebraic manipulations, but is
bringing us closer to the syntax of a programming language.

\subsubsection{Operations with parameters}
\label{sec:oper-with-param}

To motivate our second generalization, consider the theory of a module~$M$ over
a ring~$R$ (if you are not familiar with modules, think of the elements of $M$
as vectors and the elements of~$R$ as scalars). For it to be an algebraic
theory, we need to deal with scalar multiplication ${\cdot} : R \times M \to M$,
because it does not fit the established pattern. There are three possibilities:
%
\begin{enumerate}
\item We could introduce \emph{multi-sorted} algebraic theories whose operations
  take arguments from several carrier sets. The theory of a module would have
  two sorts, say $\mathsf{R}$ and $\mathsf{M}$, and scalar multiplication would
  be a binary operation of arity $(\mathsf{R}, \mathsf{M}; \mathsf{M})$. (We
  hesitate to write $\mathsf{R} \times \mathsf{M} \to \mathsf{M}$ lest the type
  theorists get useful ideas.)
\item Instead of having a single binary operation taking a scalar and a vector,
  we could have many unary operations taking a vector, one for each scalar.
\item We could view the scalar as an additional \emph{parameter} of a
  unary operation on vectors.
\end{enumerate}
%
The second and the third options are superficially similar, but they differ in
their treatment of parameters. In one case the parameters are part of the
indexing of the signature, while in the other they are properly part of the
algebraic theory. We shall adopt operations with parameters because they
naturally model algebraic operations that arise as computational effects.

\begin{example}
  The theory of a module over a ring~$(R, 0, {+}, {-}, {\cdot})$ has several
  operations. One of them is scalar multiplication, which is a \emph{unary}
  operation $\mathsf{mul}$ parameterized by element of~$R$. That is, for every
  $r \in R$ and term $t$, we may form the term
  %
  \begin{equation*}
    \mathsf{mul}(r; t),
  \end{equation*}
  %
  which we think of as~$t$ multiplied with~$r$. The remaining operations seem
  not to be parameterized, but we can force them to be parameterized by fiat.
  Addition is a binary operation $\mathsf{add}$ parameterized by the singleton
  set~$\one$: the sum of $t_1$ and $t_2$ is written as
  %
  \begin{equation*}
    \mathsf{add}(\unit; t_1, t_2).
  \end{equation*}
  %
  We can use this trick in general: an operation without parameters is an
  operation taking parameters from the singleton set.
\end{example}

Note that in the previous example we mixed theories and models. We spoke about
the \emph{theory} of a module with respect to a \emph{specific ring}~$R$.

\begin{example}
  The theory of a $\Cinfty$-ring, cf.\ Example~\ref{ex:Cinfty-theory}, may be
  reformulated using parameters. For every $n \in \NN$ there is an $n$-ary
  operation symbol $\mathsf{app}_n$ and whose parameter set is
  $\Cinfty(\RR^n, \RR)$. What was written as
  %
  \begin{equation*}
    \op{f}(t_1, \ldots, t_n)
  \end{equation*}
  %
  in Example~\ref{ex:Cinfty-theory} is now written as
  %
  \begin{equation*}
    \mathsf{app}_n(f; t_1, \ldots, t_n).
  \end{equation*}
  %
  The notation tells us what $\Cinfty$-rings are about: they are rings whose
  elements can feature as arguments to smooth functions. In contrast, an
  ordinary (commutative unital) ring is one whose elements can feature as
  arguments to ``finite degree'' smooth maps, i.e., the polynomials.

  If you insist on the~$\lambda$-notation, replace the tuple
  $(t_1, \ldots, t_n)$ of terms with a single function $t$ mapping from $[n]$ to
  terms, and write $\mathsf{app}_n(f; t)$.
\end{example}


\subsection{Algebraic theories with parameterized operations and general arities}
\label{sec:algebr-theor-with}

Let us restate the definitions of signatures and algebraic operations, with the
generalizations incorporated. For simplicity we work with sets and functions,
and leave consideration of other categories for another occasion.

A \emph{signature $\Sigma$} is given by a collection
$\family{(\op{i}{P_i}{A_i})}{i}$ of \emph{operation symbols $\op{i}$} with
associated \emph{parameter sets $P_i$} and \emph{arities~$A_i$}. The symbols may
be anything, although we think of them as syntactic entities, while $P_i$'s and
$A_i$'s are sets.

The \emph{well-founded trees $\Tree{\Sigma}{X}$} over~$\Sigma$ generated by a
set~$X$ form a set defined inductively as follows:
%
\begin{enumerate}
\item $\leaf{x} \in \Tree{\Sigma}{X}$ for every generator $x \in X$,
\item if $p \in P_i$ and $\kappa : A_i \to \Tree{\Sigma}{X}$ then
  $\op{i}(p, \kappa) \in \Tree{\Sigma}{X}$.
\end{enumerate}
%
We equate \emph{$\Sigma$-terms in context $x_1, \ldots, x_k$} with the trees
$\Tree{\Sigma}{\set{x_1, \ldots, x_k}}$.

An \emph{interpretation $I$ of a signature $\Sigma$} is given by:
%
\begin{enumerate}
\item a carrier set $\carrier{I}$,
\item for each operation symbol $\op{i}$ with parameter set~$P_i$ and arity~$A_i$,
  a map
  %
  \begin{equation*}
    \sem{\op{i}}_I : P_i \times \carrier{I}^{A_i} \longrightarrow \carrier{I}.
  \end{equation*}
\end{enumerate}
%
The interpretation $I$ may be extended to terms in contexts. A term
$x_1, \ldots, x_k \mid t$ is interpreted as a map
%
\begin{equation*}
  \sem{x_1, \ldots, x_k \mid t}_I : \carrier{I}^k \to \carrier{I}
\end{equation*}
%
as follows:
%
\begin{enumerate}
\item the variable $x_i$ is interpreted as the $i$-th projection,
  %
  \begin{equation*}
    \sem{x_0, \ldots, x_{k-1} \mid  x_i}_I = \pi_i : \carrier{I}^k \to \carrier{I},
  \end{equation*}
\item a compound term in context
  %
  \begin{equation*}
    x_1, \ldots, x_k \mid \op{i}(p, \kappa)
  \end{equation*}
  %
  is interpreted as the map
  %
  \begin{align*}
    \sem{\op{i}(p, \kappa)}_I &: \carrier{I}^k \longrightarrow \carrier{I} \\
    \sem{\op{i}(p, \kappa)}_I &:
      \eta \mapsto
      \sem{\op{i}}_I(p, \lam{a \in A_i} \sem{\kappa(a)}_I(\eta)),
  \end{align*}
\end{enumerate}

A \emph{$\Sigma$-equation} is a pair of $\Sigma$-terms $\ell$ and $r$ in context
$x_1, \ldots, x_k$, written
%
\begin{equation*}
  x_1, \ldots, x_k \mid \ell = r.
\end{equation*}
%
Given an interpretation $I$ of signature $\Sigma$, we say that such an equation
is \emph{valid} for~$I$ when the interpretations of $\ell$ and $r$ give the same
map.

An \emph{algebraic theory $\theory{T} = (\signature{T}, \equations{T})$} is
given by a signature $\signature{T}$ and a collection of $\Sigma$-equations
$\equations{T}$. A \emph{$\theory{T}$-model} is an interpretation for
$\signature{T}$ which validates all the equations~$\equations{T}$.

The notions of $\theory{T}$-morphisms and the category $\Mod{T}$ of
$\theory{T}$-models and $\theory{T}$-morphisms may be similarly generalized. We
do not repeat the definitions here, as they are almost the same. You should
convince yourself that every algebraic theory has a free model, which is still
built as a quotient of the set of well-founded trees.

\section{Computational effects as algebraic operations}
\label{sec:comp-effects-as}

It is high time we provide some examples from programming. The original insight
by Gordon Plotkin and John Power~\cite{plotkin-power} was that many
computational effects are naturally described by algebraic theories. What
precisely does this mean?

When a program runs on a computer, it interacts with the environment by
performing \emph{operations}, such as printing on the screen, reading from the
keyboard, inspecting and modifying external memory store, launching missiles,
etc. We may model these phenomena mathematically as operations on an algebra
whose elements are \emph{computations}. Leaving the exact nature of computations
aside momentarily, we note that a computation may be
%
\begin{itemize}
\item pure, in which case it terminates and returns a value, or
\item effectful, in which case it performs an operation.
\end{itemize}
%
(We are ignoring a third possibility, non-termination.) Let us write
%
\begin{equation*}
  \return{v}
\end{equation*}
%
for a pure computation that returns the value~$v$. Think of values as inert
datum that needs no further computation, such as a boolean constant, an integer,
or a $\lambda$-abstraction. An operation takes a \emph{parameter}~$p$, for
instance the memory location to be read, and a \emph{continuation}~$\kappa$,
which is a suspended computation expecting the result of the operation, for
instance the contents of the memory location that has been read. Thus it makes
sense to write
%
\begin{equation*}
  \opcall{op}{p}{\kappa}
\end{equation*}
%
for the computation that performs the operation~$\kode{op}$, with parameter~$p$
and continuation~$\kappa$. The similarity with algebraic operations from
Section~\ref{sec:algebr-theor-with} is not incidental!

\begin{example}
  The computation which increases the contents of memory location~$\ell$ by~$1$
  and returns the original contents is written as
  %
  \begin{equation*}
    \opcall{lookup}{\ell}{
    \lam{x} \opcall{update}{(\ell,x + 1)}{
    \lam{\_} \return{x}
    }
    }.
  \end{equation*}
  %
  In some venerable programming languages we would write this as $\ell{+}{+}$.
  Note that the operations happen from outside in: first the memory
  location~$\ell$ is read, its value is bound to~$x$, then $x + 1$ is written to
  memory location~$\ell$, the result of writing is ignored, and finally the
  value of~$x$ is returned.
\end{example}

So far we have a notation that looks like algebraic operations, but to do things
properly we need a signature and equations. These depend on the computational
effects under consideration. We need a notation for showing that $\kode{op}$ is an operation symbol
with parameter set~$P$ and arity~$A$. For reasons that will become clear shortly, let us use
%
\begin{equation*}
  \opdecl{\kode{op}}{P}{A}.
\end{equation*}
%
The non-standard arrow $\leadsto$ is there to remind you that $\kode{op}$ is
\emph{not} a map from $P$ to $A$.

\begin{example}
  \label{ex:theory-state}
  %
  The algebraic theory of \emph{state} is given by a set $L$ of
  \emph{locations}, a set~$S$ of stored values, and two operations
  %
  \begin{equation*}
    \opdecl{\kode{lookup}}{L}{S}
    \qquad\text{and}\qquad
    \opdecl{\kode{update}}{L \times S}{\one}.
  \end{equation*}
  %
  First we have equations which state what happens on successive lookups and updates
  to the same memory location~$\ell$:
  %
  \begin{align*}
    \opcall{lookup}{\ell}{
      \lam{s}{
        \opcall{lookup}{\ell}{
        \lam{t} \kappa \, s \, t}
      }
    } &=
    \opcall{lookup}{\ell}{\lam{s}{\kappa \, s \, s}}
    \\
    \opcall{lookup}{\ell}{
      \lam{s} \opcall{update}{(\ell, s)}{\kappa}
    } &=
    \kappa \, ()
    \\
    \opcall{update}{(\ell, s)}{
      \lam{\_} \opcall{lookup}{\ell}{\kappa}
    } &=
    \opcall{update}{(\ell, s)}{\lam{\_} \kappa \, s}
    \\
    \opcall{update}{(\ell, s)}{
      \lam{\_} \opcall{update}{(\ell, t)}{\kappa}
    } &=
    \opcall{update}{(\ell, t)}{\kappa}
  \end{align*}
  %
  For example, the first equations says that two consecutive lookups from a memory
  location give equal results. We also have equations stating that lookups and
  updated from \emph{different} locations $\ell \neq \ell'$ commute:
  %
  \begin{align*}
    \opcall{lookup}{\ell}{
       \lam{s} \opcall{lookup}{\ell'}{\lam{s'} \kappa \, s \, s'}
    } &=
    \opcall{lookup}{\ell'}{
       \lam{s'} \opcall{lookup}{\ell}{\lam{s} \kappa \, s \, s'}
    }
    \\
    \opcall{update}{(\ell, s)}{
       \lam{\_} \opcall{lookup}{\ell'}{\kappa}
    } &=
    \opcall{lookup}{\ell'}{
       \lam{t} \opcall{update}{(\ell, s)}{
          \lam{\_} \kappa \, t
       }
    } \\
    \opcall{update}{(\ell, s)}{
       \lam{\_} \opcall{update}{(\ell', s')}{\kappa}
    } &=
    \opcall{update}{(\ell', s')}{
       \lam{\_} \opcall{update}{(\ell, s)}{\kappa}
    }.
  \end{align*}
  %
  Have we forgotten any equations?
  It turns out that the theory is Hilbert-Post complete: if we add any equation
  that does not already follow from these, the theory trivializes in the sense
  that all equations become derivable.
\end{example}

\begin{example}
  \label{ex:theory-io}
  %
  The theory of \emph{input and output} (I/O) has operations
  %
  \begin{equation*}
    \opdecl{\kode{print}}{S}{\one}
    \qquad\text{and}\qquad
    \opdecl{\kode{read}}{\one}{S},
  \end{equation*}
  %
  where $S$ is the set of entities that are read or written, for example bytes,
  or strings. There are no equations. We may now write the obligatory hello world:
  %
  \begin{equation*}
    \opcall{print}{\text{`Hello world!'}}{\lam{\_} \return{\unit}}.
  \end{equation*}
\end{example}

\begin{example}
  \label{ex:theory-exception}
  %
  The theory of a point set, cf.\ Example~\ref{ex:pointed-set}, is the theory of
  an \emph{exception}. The point $\bullet$ is a constant, which takes on the
  form of a nullary operation
  %
  \begin{equation*}
    \opdecl{\kode{abort}}{\one}{\emptyset}.
  \end{equation*}
  %
  There are no equations. For example, the computation
  %
  \begin{equation*}
    \opcall{read}{\unit}{
      \lam{x}{
        \cond{x < 0}{\opcall{abort}{\unit}{\mathsf{O}_\ZZ}}{\return{(x+1)}}
      }
    }
  \end{equation*}
  %
  reads an integer~$x$ from standard input, rises an exception if $x$ is
  negative, otherwise it returns its successor.
\end{example}

\begin{example}
  Let us take the theory of semilattice, cf.\ Example~\ref{ex:semi-lattice}, but without the
  unit. It has a binary operation~$\vee$ satisfying
  %
  \begin{align*}
    x \vee x &= x, \\
    x \vee y &= y \vee x, \\
    (x \vee y) \vee z &= x \vee (y \vee z).
  \end{align*}
  %
  This is the algebraic theory of (one variant of) \emph{non-determinism}.
  Indeed, the binary operation $\vee$ corresponds to a choice operation
  %
  \begin{equation*}
    \opdecl{\kode{choose}}{\one}{\bool}
  \end{equation*}
  %
  which (non-deterministically) returns a bit, or chooses a computation, depending
  on how we look at it. Written in continuation notation, it chooses a bit~$b$
  and passes it to the continuation~$\kappa$,
  %
  \begin{equation*}
    \opcall{choose}{\unit}{\lam{b}{\kappa \, b}},
  \end{equation*}
  %
  whereas with the traditional notation it chooses between two computations
  $\kappa_0$ and $\kappa_1$,
  %
  \begin{equation*}
    \kode{choose}(\kappa_0, \kappa_1).
  \end{equation*}
  %
\end{example}

\begin{example}
  \label{ex:theory-single-state}
  %
  Algebraic theories may be combined easily. For example, if we want a theory
  describing state and I/O we may simply adjoin the signatures and equations of
  both theories to obtain their combination.

  Sometimes we want to combine theories so that the operations between them
  interact. To demonstrate this, let us consider the theory of a single stateful
  memory location holding elements of a set $S$. The operations are
  %
  \begin{equation*}
    \opdecl{\kode{get}}{\one}{S}
    \qquad\text{and}\qquad
    \opdecl{\kode{put}}{S}{\one}.
  \end{equation*}
  %
  The equations are
  %
  \begin{align*}
    \opcall{get}{\unit}{
      \lam{s}{
        \opcall{get}{\unit}{
        \lam{t} \kappa \, s \, t}
      }
    } &=
    \opcall{get}{\unit}{\lam{s}{\kappa \, s \, s}}
    \\
    \opcall{get}{\unit}{
      \lam{s} \opcall{put}{s}{\kappa}
    } &=
    \kappa \, ()
    \\
    \opcall{put}{s}{
      \lam{\_} \opcall{get}{\unit}{\kappa}
    } &=
    \opcall{put}{s}{\lam{\_} \kappa \, s}
    \\
    \opcall{put}{s}{
      \lam{\_} \opcall{put}{t}{\kappa}
    } &=
    \opcall{put}{t}{\kappa}
  \end{align*}
  %
  These are just the first three equations from Example~\ref{ex:theory-state},
  except that we need not specify which memory location to read from. It is thus
  natural to ask how to combine many instances of the theory of a single state
  to get a theory of many states. To model $I$-many states, where for each
  $\iota \in I$ the $\iota$-th state stores elements of a set~$S_\iota$, we may
  simply combine $I$-many copies of the theory of a single state, so that for
  every $\iota \in I$ we have operations
  %
  \begin{equation*}
    \opdecl{\kode{get}_\iota}{\one}{S_\iota}
    \qquad\text{and}\qquad
    \opdecl{\kode{put}_\iota}{S_\iota}{\one},
  \end{equation*}
  %
  satisfying the above equations. In addition, we have postulate
  \emph{distributivity} laws expressing the fact that operations from
  instance~$\iota$ distribute over those of instance~$\iota'$, so long as
  $\iota \neq \iota'$:
  %
  \begin{align*}
    \opcall{get_\iota}{\unit}{
       \lam{s} \opcall{get_{\iota'}}{\unit}{\lam{s'} \kappa \, s \, s'}
    } &=
    \opcall{get_{\iota'}}{\unit}{
       \lam{s'} \opcall{get_\iota}{\unit}{\lam{s} \kappa \, s \, s'}
    }
    \\
    \opcall{put_\iota}{s}{
       \lam{\_} \opcall{get_{\iota'}}{\unit}{\kappa}
    } &=
    \opcall{get_{\iota'}}{\unit}{
       \lam{t} \opcall{put_\iota}{s}{
          \lam{\_} \kappa \, t
       }
    } \\
    \opcall{put_\iota}{s}{
       \lam{\_} \opcall{update_{\iota'}}{s'}{\kappa}
    } &=
    \opcall{update_{\iota'}}{s'}{
       \lam{\_} \opcall{put_\iota}{s}{\kappa}
    }.
  \end{align*}
  %
  The theory so obtained is similar to that of Example~\ref{ex:theory-state},
  with two important differences. First, the locations $\ell \in L$ are
  parameters of operations in Example~\ref{ex:theory-state}, whereas in the
  present case the instances $\iota \in I$ index the operations themselves.
  Second, all memory locations in Example~\ref{ex:theory-state} must store
  elements of one and the same set~$S$, whereas the present case allows a
  different set $S_\iota$ for every instance~$\iota$. The price of combining the
  two possibilities into a theory of parameterized locations, each of which stores
  its own type, is steep---dependent types.
\end{example}

\subsection{Computations are free models}
\label{sec:comp-are-free}

We have theories of computational effects, but among all the models of such
theories, which one should be considered as the most accurate one? The trivial
satisfies too many equations, namely all of them, while the free model the
least, namely only those that the theory mandates. If a theory of computational
effects truly is adequately described by its signature and equations, then the
free model ought to be the desired one.

\begin{example}
  Consider the theory $\theory{State}$ of a single state storing elements of~$S$
  from Example~\ref{ex:theory-single-state}. We expect that computations which
  use $\kode{put}$ and $\kode{get}$, and return values from a set~$V$ are
  adequately modeled by the free model $\Free{State}{V}$. As we saw in
  Section~\ref{sec:free-models}, the free model is a quotient of the set of
  trees $\Tree{\signature{State}}{V}$ by a congruence
  relation~$\approx_{\theory{State}}$. Every tree is congruent to one of the
  form
  %
  \begin{equation}
    \label{eq:state-normal-form}
    %
    \opcall{get}{\unit}{
      \lam{s} \opcall{put}{f(s)}{\lam{\_} \return{g(s)}}
    }
  \end{equation}
  %
  for some maps $f : S \to S$ and $g : S \to V$. Indeed, by applying the
  equations from Example~\ref{ex:theory-single-state}, we may contract any two
  consecutive $\kode{get}$'s to a single one, and similarly for consecutive
  $\kode{put}$'s, we may disregard a $\kode{get}$ after a $\kode{put}$, and
  cancel a~$\kode{get}$ followed by a~$\kode{put}$. Thus every tree is congruent
  to one of the four forms
  %
  \begin{gather*}
    \return{v},\\
    \opcall{get}{\unit}{\lam{s} \return{g(s)}},\\
    \opcall{put}{t}{\lam{\_} \return{v}},\\
    \opcall{get}{\unit}{
      \lam{s} \opcall{put}{f(s)}{\lam{\_} \return{g(s)}}
    },
  \end{gather*}
  %
  but the first three may be brought into the form of the fourth one:
  %
  \begin{align*}
    \return{v} &= 
    \opcall{get}{\unit}{\lam{s} \opcall{put}{s}{\lam{\_} \return{v}}},
    \\
    \opcall{get}{\unit}{\lam{s} \return{g(s)}} &=
    \opcall{get}{\unit}{\lam{s} \opcall{put}{s}{\lam{\_} \return{g(s)}}},
    \\
    \opcall{put}{t}{\lam{\_} \return{v}} &=
    \opcall{get}{\unit}{\lam{\_} \opcall{put}{t}{\lam{\_} \return{v}}}.
  \end{align*}
  %
  Therefore, the free model~$\Free{Sate}(V)$ is isomorphic to the set of
  functions
  %
  \begin{equation*}
    S \to S \times V.
  \end{equation*}
  %
  The isomorphism takes the element represented by~\eqref{eq:state-normal-form}
  to the function $\lam{s} (f(s), g(s))$. (It takes extra effort to show that
  each element is represented by unique $f$ and $g$). The inverse takes a
  function $h : S \to S \times V$ to the computation represented by the tree
  %
  \begin{equation*}
    \opcall{get}{\unit}{
      \lam{s} \opcall{put}{\pi_1(h(s))}{\lam{\_} \return{\pi_2(h(s))}}
    }.
  \end{equation*}
  %
  Functional programers will surely recognize the genesis of the state monad.
\end{example}

Let us expand on the last thought of the previous example and show, at the risk
of wading a bit deeper into category theory, that free models of an algebraic
theory~$\theory{T}$ form a monad. We describe the monad structure in the form of
a Kleisli triple, because it is familiar to functional programmers. First, we
have an endofunctor $\FreeFun{T}$ on the category of sets which takes a set~$X$
to the free model $\Free{T}{X}$ and a map $f : X \to Y$ to the unique
$\theory{T}$-homomorphism $\overline{f}$ for which the following diagram
commutes:
%
\begin{equation*}
  \xymatrix{
    {X}
    \ar[r]^{\eta_X}
    \ar[d]_{f}
    &
    **[r]{\Free{T}{X}}
    \ar[d]^{\overline{f}}
    \\
    {Y}
    \ar[r]_{\eta_Y}
    &
    **[r]{\Free{T}{Y}}
  }
\end{equation*}
%
Second, the unit of the monad is the map $\eta_X : X \to \Free{T}{X}$ which
takes an element~$x$ to the element of $\Free{T}{X}$ represented by
$\return{x}$. Third, a map $\phi : X \to \Free{T}{Y}$ is lifted to the unique map
$\lift{\phi} : \Free{T}{X} \to \Free{T}{Y}$ for which the following diagram commutes:
%
\begin{equation*}
  \xymatrix{
    {X}
    \ar[r]^{\eta_X}
    \ar[rd]_{\phi}
    &
    **[r]{\Free{T}{X}}
    \ar[d]^{\lift{\phi}}
    \\
    &
    **[r]{\Free{T}{Y}}
  }
\end{equation*}
%
Concretely, $\lift{\phi}$ is defined by recursion on
($\approx_{\theory{T}}$-equivalence classes of) trees by
%
\begin{align*}
  \lift{\phi}([\return{x}]) &= \phi(x), \\
  \lift{\phi}([\opcall{op}{p}{\kappa}]) &= [\opcall{op}{p}{\lift{\phi} \circ \kappa}],
\end{align*}
%
where~$\kode{op}$ ranges over the operations of~$\theory{T}$. The first equation
holds because the above diagram commutes, and the second because $\lift{\phi}$ is a
$\theory{T}$-homomorphism. We leave the verification of the monad laws as
exercise.

\begin{example}
  Let us resume the previous example. If there's any justice, the monad for
  $\FreeFun{State}$ should be isomorphic to the usual state monad
  $(T, \theta, {}^{*})$, given by
  %
  \begin{align*}
    T(X) &= (S \to S \times X), \\
    \theta_X(x) &= (\lam{s} (s, x)), \\
    \psi^{*}(h) &= (\lam{s} \psi (\pi_2 (h(s))) (\pi_1 (h(s)))),
  \end{align*}
  %
  where $x \in X$, $\psi : X \to T(Y)$, and $h : S \to S \times X$. In the
  previous example we already verified that $\FreeFun{State}(X) \cong T(X)$ by
  the isomorphism
  %
  \begin{equation*}
    [\opcall{get}{\unit}{\lam{s} \opcall{put}{f(s)}{\lam{\_} \return{g(s)}}}]
    \mapsto
    (\lam{s} (f(s), g(s))).
  \end{equation*}
  %
  Checking that the this isomorphism transfers $\eta$ to $\theta$ and
  $\lift{{}}$ to ${}^{*}$ requires a tedious but straightforward calculation
  which is best done in the privacy of one's notebook. Nevertheless, here it is.
  Note that
  %
  \begin{equation*}
    \eta_X(x) = [\return{x}] = [\opcall{get}{\unit}{\lam{s} \opcall{put}{s}{\lam{\_} \return{x}}}]
  \end{equation*}
  %
  hence $\eta_X(x)$ is isomorphic to the map $x \mapsto (\lam{s} (s, x))$, which is
  just $\theta_X(x)$, as required. For lifting, consider any $\phi : X \to \Free{State}{Y}$.
  There corresponds to it a unique map $\psi : X \to (S \to S \times Y)$ satisfying
  %
  \begin{equation*}
    \phi(x) = [\opcall{get}{\unit}{\lam{t} \opcall{put}{\pi_1(\psi(x)(t))}{\lam{\_} \return{\pi_2(\psi(x)(t)}}}].
  \end{equation*}
  %
  First we compute $\lift{\phi}$ applied to an arbitrary element in the free model:
  %
  \begin{align*}
    \lift{\phi}&([\opcall{get}{\unit}{\lam{s} \opcall{put}{f(s)}{\lam{\_} \return{g(s)}}}]) = \\
    &[\opcall{get}{\unit}{\lam{s} \opcall{put}{f(s)}{\lam{\_} \phi(g(s))}}] = \\
    &[\opcall{get}{\unit}{\lam{s} \opcall{put}{f(s)}{\lam{\_}
      \opcall{get}{\unit}{\lam{t} \opcall{put}{\pi_1 (\psi(g(s))(t))}{\lam{\_} \return{\pi_2 (\psi(g(s))(t))}}}
    }}] = \\
    &[\opcall{get}{\unit}{\lam{s} \opcall{put}{f(s)}{\lam{\_}
      \opcall{put}{\pi_1 (\psi(g(s))(f(s)))}{\lam{\_} \return{\pi_2 (\psi(g(s))(f(s)))}}
    }}] = \\
    &[\opcall{get}{\unit}{\lam{s}
      \opcall{put}{\pi_1 (\psi(g(s))(f(s)))}{\lam{\_} \return{\pi_2 (\psi(g(s))(f(s)))}}
    }].
  \end{align*}
  %
  Then we compute $\psi^{*}$ applied to the corresponding element of the state monad:
  %
  \begin{align*}
    \psi^{*}(\lam{s} (f(s), g(s)))
    &= (\lam{s} \psi(g(s))(f(s))) \\
    &= (\lam{s} (\pi_1 (\psi(g(s))(f(s))), \pi_2 (\psi(g(s))(f(s))))),
  \end{align*}
  %
  And we have a match.
\end{example}


\subsection{Sequencing and generic operations}
\label{sec:sequ-gener-oper}

We seem to have a good theory of computations, but our notation is an
abomination which neither mathematicians nor programmers would ever want to use.
Let us provide a better syntax that will make half of them happy.

Consider an algebraic theory~$\theory{T}$. For an operation $\opdecl{op}{P}{A}$
in $\signature{T}$, define the corresponding \emph{generic operation}
%
\begin{equation*}
  \opgen{op}{p} \defeq \opcall{op}{p}{\lam{x} \return{x}}.
\end{equation*}
%
In words, the generic version performs the operation and returns its result.
When the parameter is the unit we write $\opgen{op}{}$ instead of the silly
looking $\opgen{op}{\unit}$. After a while one also grows tired of the over-line
and simplifies the notation to just $\kode{op}(p)$, but we shall not do so here.

Next, we give ourselves a better notation for the monad lifting. Suppose
$t \in \Free{T}(X)$ and $h : X \to \Free{T}{Y}$. Define the \emph{sequencing}
%
\begin{equation*}
  \seq{x}{t} h(x),
\end{equation*}
%
to be an abbreviation for $\lift{h}(t)$, with the proviso that $x$ is bound.
Generic operations and sequencing allow us to replace the awkward looking
%
\begin{equation*}
  \opcall{op}{p}{\lam{x} t(x)}
\end{equation*}
%
with
%
\begin{equation*}
  \seq{x}{\opgen{op}{p}} t(x).
\end{equation*}
%
Even better, nested operations
%
\begin{equation*}
  \opcall{op_1}{p_1}{\lam{x_1}
  \opcall{op_2}{p_2}{\lam{x_2}
  \opcall{op_3}{p_2}{\lam{x_3}
  \cdots
  }}}
\end{equation*}
%
may be written in Haskell-like notation
%
\begin{align*}
  &\seq{x_1}{\xopgen{\kode{op}_1}{p_1}} \\
  &\seq{x_2}{\xopgen{\kode{op}_3}{p_3}} \\
  &\seq{x_2}{\xopgen{\kode{op}_3}{p_3}}
  \cdots
\end{align*}
%
The syntax of a typical programming language only ever exposes the generic
operations. The generic operation $\overline{\kode{op}}$ with parameter set~$P$
and arity~$A$ looks to a programmer like a function of type $P \to A$, which is
why we use the notation $\opdecl{\kode{op}}{P}{A}$ to specify signatures.

Because sequencing is just lifting in disguise, it is governed by the same
equations as lifting:
%
\begin{align*}
  (\seq{x}{\return{v}} h(x)) &= h(v), \\
  (\seq{x}{\opcall{op}{p}{\kappa}} h(x)) &=
  \opcall{op}{p}{\lam{y} \seq{x}{\kappa(y)} h(x)}.
\end{align*}
%
These allow us to eliminate sequencing from any expression. When we rewrite the
second equation with generic operations we get an associativity law for
sequencing:
%
\begin{equation*}
  (\seq{x}{(\seq{y}{\opgen{op}{p}} \kappa(y))} h(x)) =
  (\seq{y}{\opgen{op}{p}} \seq{x}{\kappa(y)} h(x)).
\end{equation*}

The ML aficionados may be pleased to learn that the sequencing notation in
an ML-style language is none other than $\kode{let}$-binding,
%
\begin{equation*}
  \kode{let}\; x = t\;\kode{in}\;h(x).
\end{equation*}


\section{Handlers}
\label{sec:handlers}

So far the main take away is that computations returning values from~$V$ and
performing operations of a theory~$\theory{T}$ are the elements of the free
model $\Free{T}{X}$. What about \emph{transformations} between computations,
what are they? An easy but useless answer is that they are just maps between the
carriers of free models,
%
\begin{equation*}
  \carrier{\Free{T}{X}} \longrightarrow \carrier{\Free{T'}{X'}},
\end{equation*}
%
whereas a better answer should take into account the algebraic structure. Having
put so much faith in algebra, let us continue to do so and postulate that a
transformation between computations be a homomorphism. Should it be a
homomorphism with respect to~$\theory{T}$ or~$\theory{T}'$? We could weasel out
of the question by considering only homomorphisms of the form
$\Free{T}{X} \to \Free{T}{X'}$, but such homomorphisms are rather uninteresting,
because they amount to the same thing as map~$X \to \Free{T}{X'}$. We want
transformation between computations that transform the operations as well as
values.

To get a reasonable notion of transformation, let us recall that the universal
property of free models speaks about maps \emph{from} a free model. Thus, a
transformation between computations should be a $\theory{T}$-homomorphism
%
\begin{equation*}
  H : \carrier{\Free{T}{X}} \longrightarrow \carrier{\Free{T'}{X'}}.
\end{equation*}
%
For this to make any sense, the carrier $\carrier{\Free{T'}{X'}}$ must carry the
structure of a $\theory{T}$-model, i.e., in addition to~$H$ we must also
provide a $\theory{T}$-model on $\carrier{\Free{T'}{X'}}$. If we take into account the
fact that $H$ is uniquely determined by its action on the generators, we
arrive at the following notion, which we call a handler.

A \emph{handler} from computations $\Free{T}{X}$ to computations $\Free{T'}{X'}$
is given by the following data:
%
\begin{enumerate}
\item a map $f : X \to \carrier{\Free{T'}{X'}}$,
\item for every operation $\opdecl{\op{i}}{P_i}{A_i}$ in $\signature{T}$, a map
  %
  \begin{equation*}
    h_i : P_i \times \carrier{\Free{T'}{X'}}^{A_i} \to \carrier{\Free{T'}{X'}}
  \end{equation*}
  %
  such that
\item the maps $h_i$ form a $\theory{T}$-model on~$\carrier{\Free{T'}{X'}}$, i.e., they
 validate the equations~$\equations{T}$.
\end{enumerate}
%
The map $H : \carrier{\Free{T}{X}} \longrightarrow \carrier{\Free{T'}{X'}}$ induced by these
data is the unique one satisfying
%
\begin{align*}
  H([\return{x}]) &= f(x), \\
  H([\opcall{op}{p}{\kappa}]) &= h_i(p, H \circ \kappa).
\end{align*}
%
We write
%
\begin{equation*}
  H : \Free{T}{X} \hto \Free{T'}{X'}
\end{equation*}
%
when $H$ is a handler from $\Free{T}{X}$ to $\Free{T'}{X'}$.

From a mathematical point of view handlers are just a fairly unnatural
combination of algebraic notions, but they are much more interesting from a
programming point of view, as we shall demonstrate next.

We need a notation for handlers that neatly collects its defining data. Let us write
%
\begin{equation}
  \label{eq:handler-notation}
  %
  \handler \{
    \retclause{x} f(x), \;
    \big( \opclause{\op{i}}{y}{\kappa} h_i(y, \kappa) \big)_{\op{i} \in \signature{T}}
  \}
\end{equation}
%
for the handler~$H$ determined by the maps $f$ and $h_i$, as above, and
%
\begin{equation*}
  \withhandle{H}{C}
\end{equation*}
%
for the application of handler~$H$ to a computation~$C$. The defining
equations for handlers written in the new notation are, where $H$ stands for the
handler~\eqref{eq:handler-notation}:
%
\begin{align*}
  (\withhandle{H}{\return v}) &= f(v), \\
  (\withhandle{H}{\seq{x}{\xopgen{\op{i}}{p}} \kappa(x)}) &=
  h_i (p, \lam{x} \withhandle{H}{\kappa(x)})
\end{align*}

\begin{example}
  Let us consider the theory $\theory{Exn}$ of an exception, cf.\
  Example~\ref{ex:theory-exception}. A handler
  %
  \begin{equation*}
    H : \Free{Exn}(X) \hto \Free{T}{Y}
  \end{equation*}
  %
  is given by a $\kode{return}$ clause and an $\kode{abort}$ clause,
  %
  \begin{equation*}
    \handler \{
      \retclause{x} f(x),
      \opclause{\kode{abort}}{y}{\kappa} c
    \},
  \end{equation*}
  %
  where $f : X \to \Free{T}{Y}$ and $c \in \Free{T}{Y}$. Note that $c$ does not
  depend on $y \in \one$ and $\kappa : \emptyset \to \Free{T}{Y}$ because they
  are both useless. The theory of an exception has no equations, so such a
  handler is automatically well defined. Such a handler is quite similar to
  exception handlers from mainstream programming languages, except that it
  handles both the exception and the return value.
\end{example}

We could now give many examples of handlers to show that they are a very useful
programming concept which unifies and generalizes a number of techniques:
exception handlers, backtracking and other search strategies, I/O redirection,
transactional memory, coroutines, cooperative multi-threading, delimited
continuations, probabilistic programming, and many others. As this note is
already getting quite long, we recommend existing
tutorials~\cite{handler-tutorials}.

There is a programming language design question regarding handlers which we have
to answer before we can design a programming language with handlers.
For~\eqref{eq:handler-notation} to define a handler
$\Free{Exn}(X) \hto \Free{T}{Y}$, the operation clauses $h_i$ must satisfy the
equations of~$\theory{T}$. In general it is impossible to check algorithmically
whether this is the case, and so a compiler or a language interpreter should
avoid even trying to. Existing languages with handlers solve the problem by
ignoring the equations. This is not as bad as it sounds, because in practice we
often want handlers that break the equations. Moreover, dropping equations just
means that we work with trees as representatives of their equivalence classes,
which is a common implementation technique (for instance, when we represent
finite sets by lists).


\section{Making a programming language}
\label{sec:making-progr-lang}


\begin{figure}[ht]
  \centering
  \parbox{0.75\textwidth}{
    \newcommand{\bnfis}{\mathrel{\;{:}{:}\!=}\;}
    \newcommand{\bnfor}{\mathrel{\;\big|\;}}
    \begin{align*}
      \text{Value}\ v
        \bnfis& x                                & & \text{variable} \\
        \bnfor& \kode{false} \bnfor \kode{true}  & & \text{boolean constant} \\
        \bnfor& \fun{x} c                        & & \text{function} \\
        \bnfor&
          \handler \{
          \begin{aligned}[t]
            & \retclause{x} c_r, \\
            & \opclause{\op{1}}{x}{k} c_1, \ldots, \opclause{\op{n}}{x}{k} c_n\}
          \end{aligned}
          & & \text{handler}
      \\
      \text{Computation}\ c
        \bnfis& \return{v}                       & & \text{return a value} \\
        \bnfor& \opcall{\op{i}}{v}{y}{c}         & & \text{operation call} \\
        \bnfor& \seq{x}{c_1} c_2                 & & \text{sequencing} \\
        \bnfor& \conditional{v}{c_1}{c_2}        & & \text{conditional} \\
        \bnfor& v_1 \, v_2                       & & \text{application} \\
        \bnfor& \withhandle{v}{c}                & & \text{handle a computation}
    \end{align*}
  }
  \caption{Syntax of miniEff}
  \label{fig:mini-eff}
\end{figure}

\section{What is coalgebraic about algebraic effects and handlers?}

Besides strong connection with algebras, algebraic effects also have some relatively unexplored coalgebraic features. The main objects of our study will be \emph{comodels} of an algebraic theory. Let us start with an abstract categorical definition and work back from there to a useful characterization. Those not interested in categorical manipulations may skip to Section~\ref{sec:comodels} and take that as a starting point.

\subsection{Dualizing models of algebraic theories}
\label{sec:dualizing-models}

Recall that for a given category $\category{C}$, the dual category $\opcat{\category{C}}$ has the same objects as $\category{C}$, whereas its morphisms go in the other direction: morphisms between $A$ and $B$ in $\opcat{\category{C}}$ correspond exactly to morphisms $B \to A$ in $\category{C}$, which is why we will write them as $A \from B$, keeping objects in the same place and reversing the arrows. We use bold letters to write $\text{\textbf{op}}$, the standard notation for a dual category, in order to distinguish it from $\op{}$, the standard notation for an algebraic operation.

Take $\category{C}$ to be any category with finite \emph{coproducts}. We roughly follow Plotkin \& Power~\cite{power} and define the category of $\theory{T}$-comodels $\ComodC{C}{T}$ and $\theory{T}$-cohomomorphisms to be $\opcat{(\ModC{\opcat{C}}{T})}$, i.e. the dual of $\theory{T}$-models in the dual of $\category{C}$. Let us express these in more familiar terms.

First, consider the objects of $\ComodC{C}{T}$. Since a category has the same objects as its dual, these correspond to objects of $\ModC{\opcat{C}}{T}$, which are given by:
%
\begin{enumerate}
\item An object $|W|$ in $\opcat{\category{C}}$, called the \emph{carrier}. Again, a category has the same objects as its dual, these correspond to objects $|W|$ in $\category{C}$.
\item For each operation symbol $\op{i}$ a morphism in $\opcat{\category{C}}$
  %
  \begin{equation*}
    \sem{\op{i}}_W : \underbrace{|W| \times \cdots \times |W|}_{\arity{i}} \from |W|,
  \end{equation*}
  which validates all the equations of the theory (we will return to this in a moment). However, products in $\opcat{\category{C}}$ correspond to coproducts in $\category{C}$ (which is why we required $\category{C}$ to have finite coproducts), thus the above family of morphisms in $\opcat{\category{C}}$ corresponds to the family of morphisms
  %
    \begin{equation*}
    \sem{\op{i}}^W : |W| \to \underbrace{|W| + \cdots + |W|}_{\arity{i}},
  \end{equation*}
  in $\category{C}$, which we call \emph{cooperations}. We abbreviate an $n$-ary coproduct $|W| + \cdots + |W|$ by $n \cdot |W|$.
\end{enumerate}
Since the category $\ComodC{C}{T} = \opcat{(\ModC{\opcat{C}}{T})}$ has the same objects as its dual $\ModC{\opcat{C}}{T}$, such objects are exactly $\theory{T}$-comodels.

A family of cooperations on a model validates the equations of the theory~$\theory{T}$ if for every equation
%
\begin{equation*}
  x_1, \ldots, x_k \mid \ell = r
\end{equation*}
%
in~$\equations{T}$, the maps $\sem{x_1, \ldots, x_k \mid \ell}^W$ and $\sem{x_1, \ldots, x_k \mid r}^W$ are equal. Interpretations of $\Sigma$-terms are, of course, defined dually. A $\Sigma$-term in context
%
\begin{equation*}
  x_0, \ldots, x_{k-1} \mid t
\end{equation*}
%
is interpreted by a map
%
\begin{equation*}
  \sem{x_0, \ldots, x_{k-1} \mid t}^I : |I| \to k \cdot |I|,
\end{equation*}
%
as follows:
%
\begin{enumerate}
\item the variable $x_i$ is interpreted as the $i$-th injection,
  %
  \begin{equation*}
    \sem{x_0, \ldots, x_{k-1} \mid  x_i}^I = \iota_i : |I| \to k \cdot |I|,
  \end{equation*}
\item a compound term in context
  %
  \begin{equation*}
    x_0, \ldots, x_{k-1} \mid \op{i}(t_1, \ldots, t_{\arity{i}})
  \end{equation*}
  %
  is interpreted as the composition of maps
  %
  \begin{equation*}
    \xymatrix@+6em{
      {|I|} \ar[r]^{\sem{\op{i}}^I}
      &
      {{\arity{i}} \cdot |I|} \ar[r]^{[\sem{t_0}_I, \ldots, \sem{t_{\arity{i}}}_I]}
      &
      {k \cdot |I|}
    }
  \end{equation*}
\end{enumerate}
%
where for morphisms $f_i \colon A_i \to B$, we define the morphism $[f_1, \dots, f_k] \colon A_1 + \cdots + A_k \to B$ by $[f_1, \dots, f_k](\iota_j(x)) = f_j(x)$.

Finally, morphisms between comodels $W$ and $V$ in $\ComodC{C}{T}$ are exactly $\theory{T}$-homomorphisms between $V$ and $W$ in its dual $\ModC{\opcat{C}}{T}$. These in turn correspond to morphisms $\phi \colon |V| \from |W|$ in $\opcat{C}$ that commute with operations: for every operation symbol $\op{i}$ of~$\theory{T}$, we have
%
\begin{equation*}
  \phi \circ \sem{\op{i}}_V = \sem{\op{i}}_W \circ \underbrace{(\phi, \ldots, \phi)}_{\arity{i}}
\end{equation*}
in the category $\opcat{C}$. If we state this in terms of $\category{C}$, we get a morphism $\phi \colon |W| \to |V|$ that commutes with cooperations: for every operation symbol $\op{i}$ of~$\theory{T}$, we have
%
\begin{equation*}
  \sem{\op{i}}_V \circ \phi = \underbrace{[\phi, \ldots, \phi]}_{\arity{i}} \circ \sem{\op{i}}_W.
\end{equation*}

Let us now move to the category of sets, and while doing this, let us also switch to operations with parameters and a general arity. We see that in sets, $k \cdot |W| = [k] \times |W|$, so an operation symbol $\op{}$ with parameter set~$P$ and arity~$A$ can be interpreted as a cooperation $P \cdot |W| \to A \cdot |W|$.

\subsection{Comodels of algebraic theories}
\label{sec:comodels}

A \emph{cointerpretation $I$ of a signature $\Sigma$} is given by:
%
\begin{enumerate}
\item a carrier set $|I|$,
\item for each operation symbol $\op{i}$ with parameter set~$P_i$ and arity~$A_i$,
  a map
  %
  \begin{equation*}
    \sem{\op{i}}^kI : P_i \times |I| \longrightarrow A_i \times |I|,
  \end{equation*}
  which we call a \emph{cooperation}.
\end{enumerate}
%
The interpretation $I$ may be extended to terms in contexts. A term
$x_1, \ldots, x_k \mid t$ is interpreted as a map
%
\begin{equation*}
  \sem{x_1, \ldots, x_k \mid t}_I : |I| \to k \cdot |I|
\end{equation*}
%
as follows (for those that skipped the first section, $k \cdot |I| = \underbrace{|I| + \cdots + |X|}k$):
%
\begin{enumerate}
\item the variable $x_i$ is interpreted as the $i$-th injection,
  %
  \begin{equation*}
    \sem{x_0, \ldots, x_{k-1} \mid  x_i}_I = \iota_i : |I| \to k \cdot |I|,
  \end{equation*}
\item a compound term in context
  %
  \begin{equation*}
    x_1, \ldots, x_k \mid \op{i}(p, \kappa)
  \end{equation*}
  %
  is interpreted as the map
  %
  \begin{align*}
    \sem{\op{i}(p, \kappa)}_I &: |I| \longrightarrow k \cdot |I| \\
    \sem{\op{i}(p, \kappa)}_I &:
      w \mapsto (a, \kappa)
      \sem{\op{i}}_I(p, \lam{a \in A_i} \sem{\kappa(a)}_I(\eta)),
  \end{align*}
\end{enumerate}

A \emph{$\Sigma$-equation} is a pair of $\Sigma$-terms $\ell$ and $r$ in context $x_1, \ldots, x_k$, written
%
\begin{equation*}
  x_1, \ldots, x_k \mid \ell = r.
\end{equation*}
%
Given an cointerpretation $I$ of signature $\Sigma$, we say that such an equation is \emph{valid} for~$I$ when the interpretations of $\ell$ and $r$ give the same map.

An \emph{algebraic theory $\theory{T} = (\signature{T}, \equations{T})$} is
given by a signature $\signature{T}$ and a collection of $\Sigma$-equations
$\equations{T}$. A \emph{$\theory{T}$-comodel} is a cointerpretation for
$\signature{T}$ which validates all the equations~$\equations{T}$.

\begin{example}
Take the signature~$\Sigma$ with a binary operation symbol $\mathsf{choose}$. A cointerpretation of $\Sigma$ is given by a set $S$ together with a map $\sem{\mathsf{choose}}^S \colon S \to 2 \times S$. This can be seen as a boolean stream, where on each call $\sem{\mathsf{choose}}^S$ produces a boolean and the remainder of the stream.
\end{example}

\begin{example}
A state mapping locations from $L$ into values from $V$ can be described by a signature~$\Sigma$ consisting of two operation symbols: $\mathsf{get}$ with parameter set $L$ and arity $V$, and a unary $\mathsf{set}$ with parameter set $L \times V$. The cointerpretation of $\Sigma$ amounts to a set $S$ with maps $\sem{\mathsf{get}} \colon L \times S \to V \times S$ and $\sem{\mathsf{set}} \colon (L \times V) \times S \to S$. One such set would be the set of functions $V^L$, where $\sem{\mathsf{get}}(\ell, s) = s(\ell)$ and $\sem{\mathsf{get}}((\ell, v), s) = s[\ell \mapsto v]$, i.e. a function that behaves as $s$, but maps $\ell$ to $v$.
\end{example}

\begin{example}
Output is represented with a unary operation symbol $\mathsf{print}$ with a parameter set $O$ representing the output alphabet. Cointerpretations of such signature are given by sets $W$ operation A state mapping locations from $L$ into values from $V$ can be described by a signature~$\Sigma$ consisting of two operation symbols: $\mathsf{get}$ with parameter set $L$ and arity $V$, and a unary $\mathsf{set}$ with parameter set $L \times V$. The cointerpretation of $\Sigma$ amounts to a set $S$ with maps $\sem{\mathsf{get}} \colon L \times S \to V \times S$ and $\sem{\mathsf{set}} \colon (L \times V) \times S \to S$. One such set would be the set of functions $V^L$, where $\sem{\mathsf{get}}(\ell, s) = s(\ell)$ and $\sem{\mathsf{get}}((\ell, v), s) = s[\ell \mapsto v]$, i.e. a function that behaves as $s$, but maps $\ell$ to $v$.
\end{example}



\begin{example}
Note that if $\Sigma$ contains a nullary operation symbol $\op : 0$, no nontrivial cointerpretation of $\Sigma$ exists, as that would entail a set $|W|$ and a map $|W| \to \emptyset$, which is possible only if $|W|$ is itself $\emptyset$.
\end{example}

\subsection{Using comodels for interpreting hardware}

We can think of comodels~$W$ as hardware on which operations are executed. The carrier~$|W|$ represents the state, while cooperations $|W| \times P \to |W| \times A$ take the initial state and a parameter, and produce the result and the altered state.

We can use this to execute programs, which, recall, are represented with models. Take a model~$M$ representing the software and a comodel~$W$ representing the hardware  on which it is run. Next, take an initial configuration $\conf{m}{w}$, where $m \in M$ is a program and $w \in W$ is the inital state. If $m$ is of the form $\return v$, our program has terminated with the resulting value $v$ in a final state~$w$. But if $m$ calls an operation and is of the form $\op{M}(p, \kappa)$, we can execute $\op{}$ on $W$. By applying the cooperation $\sem{\op}^W$ to the parameter $p$ and the world~$w$, we get a result~$a$ and a new world~$w'$. We pass $a$ to $\kappa$ to get a continuation, which we resume in the new state $w'$.

This relation induces an equivalence $\equiv$ on the set of configurations $M \times W$, given by:
\[
  \conf{\op{M}(p, \kappa)}{w} \equiv \conf{\kappa(a)}{w'} \quad\text{where $\op{}^W(p; w) = (a, w')$}
\]
We label the resulting quotient set $(M \times W) / \equiv$ by $M \otimes W$ and call it the \emph{tensor product}, because of its similarity to how scalars in a tensor product pass from one component to the other.

In case the theory~$\theory{T}$ is nontrivial, we must check that the relation is well-defined, for $\op{M}(p, \kappa)$ may be equivalent to some other $m' \in M$. Since $W$ is also a comodel, $\op{}^W$ obeys the same equations. We leave the proof as an exercise for the reader.

\subsection{Tensoring models and comodels of algebraic theories}
\label{sec:tensoring-models}

We have used comodels to represent the hardware that executes the operations that escape all handlers and reach the toplevel. Another frequent viewpoint is to represent it with a ``default handler'' which sits at the top and encompasses the whole executed computation. Let us compare the two representations and see why the former seems a better fit. First, we can show that any such hardware can indeed be simulated by a paramater-passing handler. For a comodel $W$ with cooperations, we can define a model $(X \times W)^W$ where the program computing values from $V$ can be ...

\bibliographystyle{plain}
\bibliography{what-is-algebraic}


\end{document}
