\documentclass{amsart}

\usepackage[T1]{fontenc}
\usepackage[utf8]{inputenc}
\usepackage{amsmath,amssymb,amsthm}
\usepackage{times}

\newcommand{\NN}{\mathbb{N}}

\newcommand{\all}[1]{\forall #1 \,.\,}
\newcommand{\some}[1]{\exists #1 \,.\,}
\newcommand{\set}[1]{\{#1\}}
\newcommand{\such}{\mid}

\newcommand{\op}{\mathsf{op}}

{\theoremstyle{definition}
\newtheorem{definition}{Definition}
\newtheorem{example}[definition]{Example}
}

\begin{document}

\title{What is algebraic about algebraic effects?}

\author{Andrej Bauer}

\begin{abstract}
  A short tutorial on algebraic theories and how they are related to the theory of
  algebraic effects and handlers.
\end{abstract}

\maketitle

\section{Algebraic theories}
\label{sec:algebraic-theories}


In algebra we study mathematical structures that are equipped with operations satisfying
equational laws. For example, a group is a structure $(G, \mathsf{u}, {\cdot}, {}^{-1})$ satisfying
the familiar group identities:
%
\begin{gather*}
  (x \cdot y) \cdot z = x \cdot (y \cdot z),\\
  \mathsf{u} \cdot x = x = x \cdot \mathsf{u},\\
  x \cdot x^{-1} = \mathsf{u} = x^{-1} \cdot x.
\end{gather*}
%
There are alternative axiomatizations of groups. For example, a group is a monoid
$(G, \mathsf{u}, {\cdot})$ in which every element is invertible,
$\all{x \in G} \some{y \in G} x \cdot y = \mathsf{u} = y \cdot x$. However, we prefer the
form in which the axioms are just equations between terms built from variables and
operations (a constant is construed as a nullary operation). Such theories are known as
\emph{algebraic} or \emph{equational theories}.


\begin{definition}
  A \emph{signature $\Sigma$} is a collection of \emph{operation symbols}
  $\op_1, \op_2, \op_3, \ldots$, each of which has an associated \emph{arity}
  $n_i \in \NN$. The \emph{$\Sigma$-terms} are built inductively using the following
  rules:
  %
  \begin{enumerate}
  \item \emph{variables} $x_0, x_1, x_2, \ldots$ are $\Sigma$-terms,
  \item if $t_1, \ldots, t_{n_i}$ are $\Sigma$-terms then $\op_i(t_1, \ldots, t_{n_i})$ is
    a $\Sigma$-term.
  \end{enumerate}
\end{definition}

It is best not to think of $\Sigma$-terms as expressions or strings of symbols, but rather
as \emph{finite trees} whose leaves are variables and whose nodes are labeled with operation
symbols. The node labeled with $\op_i$ has $n_i$ subtrees.

\begin{definition}
  An \emph{algebraic theory} is given by a signature~$\Sigma$ and a collection of equations
  between $\Sigma$-terms.
\end{definition}




\footnote{It is \emph{not} the case that the variables appearing
  in the equations ``implicitly universally quantified''. They could be, but we do not
  quantify them. Instead, these are \emph{free} variables.}






\end{document}
