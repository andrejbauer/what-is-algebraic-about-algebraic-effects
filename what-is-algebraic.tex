\documentclass{amsart}

\usepackage[T1]{fontenc}
\usepackage[utf8]{inputenc}
\usepackage{amsmath,amssymb,amsthm}
\usepackage{times}
\usepackage{xypic}

\newcommand{\NN}{\mathbb{N}}

\newcommand{\all}[1]{\forall #1 \,.\,}
\newcommand{\some}[1]{\exists #1 \,.\,}
\newcommand{\set}[1]{\{#1\}}
\newcommand{\such}{\mid}

\newcommand{\op}{\mathsf{op}}
\newcommand{\one}{\mathsf{1}}

\newcommand{\sem}[1]{[\![#1]\!]}

{\theoremstyle{definition}
\newtheorem{definition}{Definition}[section]
\newtheorem{example}[definition]{Example}
}

\begin{document}

\title{What is algebraic about algebraic effects?}

\author{Andrej Bauer}

\begin{abstract}
  This is a short tutorial on algebraic theories and their relation to computational
  effects, as known in the theory of programming languages. An unusually large number of
  examples is provided, and excursions into category theory are kept to a minimum.
\end{abstract}

\maketitle

\section{Algebraic theories}
\label{sec:algebraic-theories}


\subsection{Signatures and equations}
\label{sec:signatures-equations}

In algebra we study mathematical structures that are equipped with operations satisfying
equational laws. For example, a group is a structure $(G, \mathsf{u}, {\cdot}, {}^{-1})$,
where $\mathsf{u}$~is a constant, $\cdot$~is a binary operation, and ${}^{-1}$~is a unary
operation, satisfying the familiar group identities:
%
\begin{gather*}
  (x \cdot y) \cdot z = x \cdot (y \cdot z),\\
  \mathsf{u} \cdot x = x = x \cdot \mathsf{u},\\
  x \cdot x^{-1} = \mathsf{u} = x^{-1} \cdot x.
\end{gather*}
%
There are alternative axiomatizations of groups. For example, a group is a monoid
$(G, \mathsf{u}, {\cdot})$ in which every element is invertible, i.e., for every $x$ there
exists $y$ such that $x \cdot y = \mathsf{u} = y \cdot x$. However, we prefer the
formulation in which the axioms are just equations between terms built from variables and
operations (a constant is construed as a nullary operation). Such theories are known as
\emph{algebraic} or \emph{equational theories}.

\begin{definition}
  \label{def:algebraic-theory}
  %
  A \emph{signature $\Sigma$} is a collection of \emph{operation symbols}
  $\op_1, \op_2, \op_3, \ldots$, each of which has an associated \emph{arity}
  $n_i \in \NN$. The \emph{$\Sigma$-terms} are built inductively using the following
  rules:
  %
  \begin{enumerate}
  \item \emph{variables} $x_0, x_1, x_2, \ldots$ are $\Sigma$-terms,
  \item if $t_1, \ldots, t_{n_i}$ are $\Sigma$-terms then $\op_i(t_1, \ldots, t_{n_i})$ is
    a $\Sigma$-term.
  \end{enumerate}
  %
  An \emph{algebraic theory} is given by a signature~$\Sigma$ and a collection of equations
  between $\Sigma$-terms.
\end{definition}

An operation symbol whose arity is~$0$ is called a \emph{constant} or a \emph{nullary}
symbol. Operation symbols with arities $1$, $2$ and $3$ are referred to as \emph{unary},
\emph{binary}, and \emph{ternary}, respectively.

It is best to think of $\Sigma$-terms not as expressions, but as \emph{finite trees} whose
leaves are variables and whose nodes are labeled with operation symbols. The node labeled
with $\op_i$ has $n_i$ subtrees. Similarly, equations should be viewed as data that the
algebraic theory incoporates, rather than logical statements, i.e., $\ell = r$ is just a
suggestive way of displaying the two terms $\ell$ and $r$.

We impose no restrictions on the number of operation symbols or equations, but at least in
classical treatments of the subject certain complications are avoided by insisting that
arities be non-negative integers.

\begin{example}
  The theory of a group is algebraic. In order to follow closely
  Definition~\ref{def:algebraic-theory}, we eschew the traditional notation $\cdot$ and
  ${}^{-1}$, and use only the variables $x_0, x_1, x_2, \ldots$ We abide by such
  formalistic requirements once to demonstrate them, but shall take notational
  liberties subsequently.
  %
  The theory of a group is given by operation symbols $\mathsf{u}$, $\mathsf{m}$, and
  $\mathsf{i}$ whose arities are $0$, $2$, and $1$, respectively. The equations are:
  %
  \begin{gather*}
    \mathsf{m}(\mathsf{m}(x_0, x_1), x_2) = \mathsf{m}(x_0, \mathsf{m}(x_1, x_2)),\\
    \mathsf{m}(\mathsf{u}(), x_0) = x_0 = \mathsf{m}(x_0, \mathsf{u}()),\\
    \mathsf{m}(x_0, \mathsf{i}(x_0)) = \mathsf{u}() = \mathsf{m}(\mathsf{i}(x_0), x_0).
  \end{gather*}
  %
\end{example}

\begin{example}
  \label{ex:semi-lattice}
  %
  The theory of a semi-lattice is algebraic. It is given by a nullary symbol $\bot$ and a
  binary symbol $\vee$, satisfying the equations
  %
  \begin{align*}
    x \vee (y \vee z) &= (x \vee y) \vee z,\\
    x \vee y &= y \vee x,\\
    x \vee x &= x,\\
    x \vee \bot &= x.
  \end{align*}
\end{example}

\begin{example}
  \label{ex:field}
  %
  The theory of a field, as usually given, is not algebraic because the inverse~$0^{-1}$
  is undefined, whereas the operations of an algebraic theory are always taken to be
  total. However, a proof is required to show that there is no equivalent algebraic theory.
\end{example}

\begin{example}
  \label{ex:pointed-set}
  %
  The theory of a \emph{pointed set} has a constant $\star$ and no equations.
\end{example}

\begin{example}
  The \emph{empty theory} has no operation symbols and no equations.
\end{example}

\begin{example}
  The theory of a \emph{singleton} has a constant $\star$ and the equation $x = y$.
\end{example}

\begin{example}
  \label{ex:lattice}
  %
  A bounded lattice is a partial order with finite infima and suprema. Such a formulation
  is not algebraic because the infimum and supremum operators do not have fixed arities,
  but we can reformulate it in terms of nullary and binary operations.
  Thus, the theory of a bounded lattice has constants $\bot$ and $\top$, and two binary
  operation symbols $\vee$ and $\wedge$, satisfying the equations:
  %
  \begin{align*}
    x \vee (y \vee z) &= (x \vee y) \vee z,   &      x \wedge (y \wedge z) &= (x \wedge y) \wedge z,\\
    x \vee y &= y \vee x,                     &      x \wedge y &= y \wedge x,\\
    x \vee x &= x,                            &      x \wedge x &= x,\\
    x \vee \bot &= x,                         &      x \wedge \top &= x.
  \end{align*}
  %
  Notice that the theory of a bounded lattice is simply the juxtaposition of two copies of
  the theory of a semi-lattice from Example~\ref{ex:semi-lattice}. The partial order is
  recovered because $x \leq y$ is equivalent to $x \vee y = y$ and to $x \wedge y = x$.
\end{example}

\begin{example}
  \label{ex:finitely-generated-group}
  %
  A \emph{finitely generated group} is a group which contains a finite collection of
  elements, called the \emph{generators}, such that every element of the group is obtained
  by multiplications and inverses of the generators. It is not clear how to express this
  condition using only equations, but a proof is required to show that there is no
  equivalent algebraic theory.
\end{example}


\subsection{Models and morphisms}
\label{sec:models-and-morphisms}

Each algebraic theory describes a certain kind of mathematical structures, which we call
the \emph{models} of the theory. Let us explain more carefully how one obtains the models
from the theory.

Let a signature $\Sigma$ be given. An \emph{interpretation~$I$} of $\Sigma$ is given by
the following data:
% 
\begin{enumerate}
\item a set $S$, called the \emph{carrier},
\item for each operation symbol $\op_i$ of arity $n_i$ a map
  %
  \begin{equation*}
    \sem{\op_i}_I : \underbrace{S \times \cdots \times S}_{n_i} \to S.
  \end{equation*}
\end{enumerate}
%
We shall abbreviate an $n$-ary product $S \times \cdots \times S$ as $S^n$. A nullary
product $S^0$ is the singleton set~$\one = \{\star\}$, which matches our expectation that
a nullary operation symbol be interpreted by an element of~$S$. Indeed, the elements of
$S$ are in bijective correspondence with maps $\one \to S$.


An interpretation~$I$ may be extended to all $\Sigma$-terms. A \emph{context} is a list of
variables $x_0, \ldots, x_{k-1}$. A \emph{term in a context} is a $\Sigma$-term $t$
such that every variable appearing in $t$ is listed by the context. We write
%
\begin{equation*}
  x_0, \ldots, x_{k-1} \mid t
\end{equation*}
%
to indicate that $t$ is a $\Sigma$-term in context $x_0, \ldots, x_{k-1}$. Such a term in
context is interpreted by a map
%
\begin{equation*}
  \sem{x_0, \ldots, x_{k-1} \mid t}_I : S^k \to S,
\end{equation*}
%
as follows:
% 
\begin{enumerate}
\item the variable $x_i$ is interpreted as the $i$-th projection,
  %
  \begin{equation*}
    \sem{x_0, \ldots, x_{k-1} \mid  x_i}_I = \pi_i : S^k \to S,
  \end{equation*}
\item a compound term in context 
  %
  \begin{equation*}
    x_0, \ldots, x_{k-1} \mid \op_i(t_1, \ldots, t_{n_i})
  \end{equation*}
  %
  is interpreted as the composition of maps
  %
  \begin{equation*}
    \xymatrix@+6em{
      {S^k} \ar[r]^{(\sem{t_0}_I, \ldots, \sem{t_{n_i}}_I)}
      &
      {S^{n_i}} \ar[r]^{\sem{\op_i}_I}
      &
      {S}
    }
  \end{equation*}
  %
  where we elided the contexts $x_0, \ldots, x_{k-1}$ for the sake of brevity, and we
  shall continue to do so.
\end{enumerate}
%
A \emph{model} of an algebraic theory is an interpretation of its signature which
validates the equations. That is, if $\ell = r$ is an equation of the theory


\subsection{Variations}
\label{sec:variations}

\subsubsection{Operations with parameters}
\label{sec:oper-with-param}

\subsubsection{Multi-sorted algebraic theories}
\label{sec:multi-sort-algebr}

\subsubsection{Essentially algebraic theories}
\label{sec:essent-algebr-theor}



\section{Computational effects as algebraic operations}
\label{sec:comp-effects-as}

\section{Algebraic handlers}
\label{sec:algebraic-handlers}



\end{document}
