\documentclass{amsart}

\usepackage[T1]{fontenc}
\usepackage[utf8]{inputenc}
\usepackage[hidelinks]{hyperref}
\usepackage{amsmath,amssymb,amsthm}
\usepackage{times}
\usepackage{xypic}

\newcommand{\NN}{\mathbb{N}} % natural numbers
\newcommand{\RR}{\mathbb{R}} % real numbers
\newcommand{\ZZ}{\mathbb{Z}} % integers

\newcommand{\theory}[1]{\mathsf{#1}} % A named theory
\newcommand{\signature}[1]{\Sigma_{\theory{#1}}} % The signature of a theory
\newcommand{\equations}[1]{\mathcal{E}_{\theory{#1}}} % Equations of a theory

\newcommand{\Mod}[1]{\text{\textbf{Mod}}(\theory{#1})} % Models of a theory in Set
\newcommand{\ModC}[2]{\text{\textbf{Mod}}_{\category{#1}}(\theory{#2})} % Models of a theory in a category

\newcommand{\category}[1]{\text{\textbf{#1}}} % A named category
\newcommand{\Set}{\category{Set}} % The category of sets

\newcommand{\Free}[2]{F_{\theory{#1}}(#2)} % free model

\newcommand{\all}[1]{\forall #1 \,.\,} % universal quantifier
\newcommand{\some}[1]{\exists #1 \,.\,} % existential quantifier
\newcommand{\set}[1]{\{#1\}} % set description
\newcommand{\such}{\mid}

\newcommand{\lam}[1]{\lambda #1 \,.\,}

\newcommand{\family}[2]{\{#1\}_{#2}} % a family

\newcommand{\finpow}[1]{\mathcal{P}_{{<}\omega}(#1)} % finite powerset

\newcommand{\Tree}[2]{\mathsf{Tree}_{#1}(#2)} % trees over a signature
\newcommand{\leaf}[1]{\mathsf{leaf}(#1)} % the embedding of generators into trees

\newcommand{\op}[1]{\mathsf{op}_{#1}} % an operation symbol
\newcommand{\arity}[1]{\mathsf{ar}_{#1}} % arity of a symbol

\newcommand{\one}{\mathsf{1}} % the terminal object

\newcommand{\Cinfty}{\mathcal{C}^\infty}

\newcommand{\sem}[1]{[\![#1]\!]} % semantic bracket

\newcommand{\bool}{\mathsf{bool}} % booleans
\newcommand{\true}{\mathsf{true}}
\newcommand{\false}{\mathsf{false}}
\newcommand{\cond}[3]{\mathsf{if}\;#1\;\mathsf{then}\;#2\;\mathsf{else}\;#3}

{\theoremstyle{definition}
\newtheorem{definition}{Definition}[section]
\newtheorem{example}[definition]{Example}
}

\begin{document}

\title{What is algebraic about algebraic effects?}

\author{Andrej Bauer}

\begin{abstract}
  This is a short tutorial on algebraic theories and their relation to computational
  effects, as known in the theory of programming languages. An unusually large number of
  examples is provided, and excursions into category theory are kept to a minimum.
\end{abstract}

\maketitle

\section{Algebraic theories}
\label{sec:algebraic-theories}


In algebra we study mathematical structures that are equipped with operations satisfying
equational laws. For example, a group is a structure $(G, \mathsf{u}, {\cdot}, {}^{-1})$,
where $\mathsf{u}$~is a constant, $\cdot$~is a binary operation, and ${}^{-1}$~is a unary
operation, satisfying the familiar group identities:
%
\begin{gather*}
  (x \cdot y) \cdot z = x \cdot (y \cdot z),\\
  \mathsf{u} \cdot x = x = x \cdot \mathsf{u},\\
  x \cdot x^{-1} = \mathsf{u} = x^{-1} \cdot x.
\end{gather*}
%
There are alternative axiomatizations, for instance: a group is a monoid
$(G, \mathsf{u}, {\cdot})$ in which every element is invertible, i.e., for every $x$ there
exists $y$ such that $x \cdot y = \mathsf{u} = y \cdot x$. However, we prefer a
formulation whose axioms are equations. Such theories are known as
\emph{algebraic} or \emph{equational theories}.

\subsection{Signatures, terms and equations}
\label{sec:signatures-equations}

A \emph{signature $\Sigma$} is a collection of \emph{operation symbols} with
\emph{arities} $\family{(\op{i}, \arity{i})}{i}$. The operation symbols
$\op{i}$ may be any anything, but are usually thought of as syntactic entities,
while arities $\arity{i}$ are non-negative integers. An operation symbol whose
arity is~$0$ is called a \emph{constant} or a \emph{nullary} symbol. Operation
symbols with arities $1$, $2$ and $3$ are referred to as \emph{unary},
\emph{binary}, and \emph{ternary}, respectively.

A (possibly empty) list of distinct variables $x_1, \ldots, x_k$ is called a
\emph{context}. The \emph{$\Sigma$-terms in context $x_1, \ldots, x_k$} are
built inductively using the following rules:
% 
\begin{enumerate}
\item each variable $x_i$ is a $\Sigma$-term in context $x_1, \ldots, x_k$,
\item if $t_1, \ldots, t_{\arity{i}}$ are $\Sigma$-terms in context $x_1, \ldots, x_k$ then
  $\op{i}(t_1, \ldots, t_{\arity{i}})$ is a $\Sigma$-term in context $x_1, \ldots, x_k$.
\end{enumerate}
%
We write
%
\begin{equation*}
  x_1, \ldots, x_k \mid t
\end{equation*}
%
to indicate that $t$ is a $\Sigma$-term in the given context. A \emph{closed
  $\Sigma$-term} is a $\Sigma$-term in the empty context. No variables occur
in a closed term.

A \emph{$\Sigma$-equation} is a pair of $\Sigma$-terms $\ell$ and $r$ in context
$x_1, \ldots, x_k$. We write
%
\begin{equation*}
  x_1, \ldots, x_k \mid \ell = r
\end{equation*}
%
to indicate an equation in a context. We shall often elide the context and write simply
$\ell = r$, but it should be understood that there is an ambient context which contains at
least all the variables mentioned by $\ell$ and $r$. When no confusion can arise we also
drop the prefix ``$\Sigma$-'' and speak of just terms and equations instead of
$\Sigma$-terms and $\Sigma$-equations.

\begin{example}
  \label{ex:monoid-signature}
  %
  The signature for the theory of a monoid has a nullary symbol $\mathsf{u}$ and a binary
  symbol $\mathsf{m}$. There are infinitely many expressions in context $x, y$, such as
  %
  \begin{align*}
    \mathsf{u}(),\quad
    x,\quad
    y,\quad
    \mathsf{m}(\mathsf{u}(), \mathsf{u}()),\quad
    \mathsf{m}(\mathsf{u}(), x),\quad
    \mathsf{m}(y, \mathsf{u}()),\quad
    \mathsf{m}(x, x),\quad
    \mathsf{m}(y, x),
    \ldots
  \end{align*}
  %
  An equation in context $x, y$ is
  %
  \begin{equation*}
    x, y \mid \mathsf{m}(y, x) = \mathsf{m}(\mathsf{m}(\mathsf{u}(), x), y).
  \end{equation*}
  %
  It is customary to write a nullary symbol $\mathsf{u}()$ simply as $\mathsf{u}$, and to
  use infix an infix operator~$\cdot$ in place of~$\mathsf{m}$. With such notation the
  above equation would be written as
  %
  \begin{equation*}
    x, y \mid y \cdot x = (\mathsf{u} \cdot x) \cdot y.
  \end{equation*}
  %
  One might even omit $\cdot$ and the context, in which case the equation is written
  simply as $y \, x = (\mathsf{u} \, x) \, y$. If we agree that $\cdot$ associates to the
  left then $(\mathsf{u} \, x) \, y$ may be written as $\mathsf{u} \, x \, y$, and we are
  left with $y \, x = \mathsf{u} \, x \, y$. Note that we are \emph{not} discussing
  validity of equations but only methods for writing them down. It is irrelevant whether
  the above equation is valid in monoids.
\end{example}


\subsection{Algebraic theories}
\label{sec:algebraic-theories-1}

An \emph{algebraic theory $\theory{T} = (\signature{T}, \equations{T})$} is given by a
signature~$\signature{T}$ and a collection $\equations{T}$ of $\signature{T}$-equations.
%
We impose no restrictions on the number of operation symbols or equations, but at least in
classical treatments of the subject certain complications are avoided by insisting that
arities be non-negative integers.

\begin{example}
  \label{ex:theory-group}
  %
  The theory~$\theory{Group}$ of a group is algebraic. In order to follow closely the
  definitions we eschew the traditional notation $\cdot$ and ${}^{-1}$, and explicitly
  display the contexts. We abide by such formalistic requirements once to demonstrate
  them, but shall take notational liberties subsequently.
  %
  The signature $\signature{Group}$ is given by operation symbols $\mathsf{u}$,
  $\mathsf{m}$, and $\mathsf{i}$ whose arities are $0$, $2$, and $1$, respectively. The
  equations $\equations{Group}$ are:
  %
  \begin{align*}
    x, y, z &\mid \mathsf{m}(\mathsf{m}(x, y), z) = \mathsf{m}(x, \mathsf{m}(y, z)),\\
    x &\mid \mathsf{m}(\mathsf{u}(), x) = x \\
    x &\mid \mathsf{m}(x_0, \mathsf{u}()) = x,\\
    x &\mid \mathsf{m}(x, \mathsf{i}(x)) = \mathsf{u}()\\
    x &\mid \mathsf{m}(\mathsf{i}(x), x) = \mathsf{u}().
  \end{align*}
  %
\end{example}

\begin{example}
  \label{ex:semi-lattice}
  %
  The theory $\theory{Semilattice}$ of a semilattice is algebraic. It is given by a
  nullary symbol $\bot$ and a binary symbol $\vee$, satisfying the equations
  %
  \begin{align*}
    x \vee (y \vee z) &= (x \vee y) \vee z,\\
    x \vee y &= y \vee x,\\
    x \vee x &= x,\\
    x \vee \bot &= x.
  \end{align*}
  %
  It should be clear that the first equation has context $x, y, z$, the second one
  in~$x, y$, and the last two in~$x$.
\end{example}

\begin{example}
  \label{ex:field}
  %
  The theory of a field, as usually given, is not algebraic because the inverse~$0^{-1}$
  is undefined, whereas the operations of an algebraic theory are always taken to be
  total. However, a proof is required to show that there is no equivalent algebraic theory.
\end{example}

\begin{example}
  \label{ex:pointed-set}
  %
  The theory $\theory{Set_\bullet}$ of a \emph{pointed set} has a constant $\bullet$ and
  no equations.
\end{example}

\begin{example}
  \label{ex:theory-empty}
  %
  The \emph{empty theory $\theory{Empty}$} has no operation symbols and no equations.
\end{example}

\begin{example}
  \label{ex:theory-singleton}
  %
  The theory of a \emph{singleton $\theory{Singleton}$} has a constant $\star$ and the
  equation $x = y$.
\end{example}

\begin{example}
  \label{ex:lattice}
  %
  A bounded lattice is a partial order with finite infima and suprema. Such a formulation
  is not algebraic because the infimum and supremum operators do not have fixed arities,
  but we can reformulate it in terms of nullary and binary operations. Thus, the theory
  $\theory{Lattice}$ of a bounded lattice has constants $\bot$ and $\top$, and two binary
  operation symbols $\vee$ and $\wedge$, satisfying the equations:
  %
  \begin{align*}
    x \vee (y \vee z) &= (x \vee y) \vee z,   &      x \wedge (y \wedge z) &= (x \wedge y) \wedge z,\\
    x \vee y &= y \vee x,                     &      x \wedge y &= y \wedge x,\\
    x \vee x &= x,                            &      x \wedge x &= x,\\
    x \vee \bot &= x,                         &      x \wedge \top &= x.
  \end{align*}
  %
  Notice that the theory of a bounded lattice is simply the juxtaposition of two copies of
  the theory of a semi-lattice from Example~\ref{ex:semi-lattice}. The partial order is
  recovered because $x \leq y$ is equivalent to $x \vee y = y$ and to $x \wedge y = x$.
\end{example}

\begin{example}
  \label{ex:finitely-generated-group}
  %
  A \emph{finitely generated group} is a group which contains a finite collection of
  elements, called the \emph{generators}, such that every element of the group is obtained
  by multiplications and inverses of the generators. It is not clear how to express this
  condition using only equations, but a proof is required to show that there is no
  equivalent algebraic theory.
\end{example}

\begin{example}
  \label{ex:Cinfty-theory}
  %
  An example of an algebraic theory with many operations and equations is the theory of a
  $\Cinfty$-ring. Let $\Cinfty(\RR^n, \RR^m)$ be the set of all smooth maps from $\RR^n$
  to $\RR^m$. The signature for the theory of a $\Cinfty$-ring contains an $n$-ary
  operation symbol $\op{f}$ for each $f \in \Cinfty(\RR^n, \RR)$. For all
  $f \in \Cinfty(\RR^n, \RR)$, $h \in \Cinfty(\RR^m, \RR)$, and
  $g_1, \ldots, g_n \in \Cinfty(\RR^m, \RR)$ such that
  %
  \begin{equation*}
    f \circ (g_1, \ldots, g_n) = h,
  \end{equation*}
  %
  the theory has the equation
  %
  \begin{equation*}
    x_1, \ldots, x_m \mid
    \op{f} (\op{g_1}(x_1, \ldots, x_m), \ldots, \op{g_n}(x_1, \ldots, x_m)) =
    \op{h}(x_1, \ldots, x_m).
  \end{equation*}
  %
  The theory contains the theory of a commutative unital ring as a subtheory. Indeed,
  the ring operations on~$\RR$ are smooth maps, and so they appear as $\op{+}$,
  $\op{\times}$, $\op{-}$ in the signature, and so do constants $\op{0}$ and $\op{1}$,
  because all maps $\RR^0 \to \RR$ are smooth. The commutative ring equations are present
  as well because the real numbers form a commutative ring.
\end{example}


\subsection{Interpretations of signatures}
\label{sec:interp-of-sign}

Let a signature $\Sigma$ be given. An \emph{interpretation~$I$} of $\Sigma$ is given by
the following data:
%
\begin{enumerate}
\item a set $|I|$, called the \emph{carrier},
\item for each operation symbol $\op{i}$ a map
  %
  \begin{equation*}
    \sem{\op{i}}_I : \underbrace{|I| \times \cdots \times |I|}_{\arity{i}} \to |I|.
  \end{equation*}
\end{enumerate}
%
The double bracket $\sem{{\ }}_I$ is called the \emph{semantic bracket} and is typically
used when syntactic entities (operation symbols, terms, equations) are mapped by~$I$ to
their mathematical counterparts. When no confusion can arise, we omit the subscript~$I$
and write just~$\sem{{\ }}$.

We abbreviate an $n$-ary product $|I| \times \cdots \times |I|$ as $|I|^n$. A nullary product
$|I|^0$ is the singleton set~$\one = \{\star\}$, which matches our expectation that a
nullary operation symbol be interpreted by an element of~$|I|$. Indeed, the elements of $|I|$
are in bijective correspondence with maps $\one \to |I|$.

An interpretation~$I$ may be extended to $\Sigma$-terms. A $\Sigma$-term in context
%
\begin{equation*}
  x_0, \ldots, x_{k-1} \mid t
\end{equation*}
%
is interpreted by a map
%
\begin{equation*}
  \sem{x_0, \ldots, x_{k-1} \mid t}_I : |I|^k \to |I|,
\end{equation*}
%
as follows:
%
\begin{enumerate}
\item the variable $x_i$ is interpreted as the $i$-th projection,
  %
  \begin{equation*}
    \sem{x_0, \ldots, x_{k-1} \mid  x_i}_I = \pi_i : |I|^k \to |I|,
  \end{equation*}
\item a compound term in context
  %
  \begin{equation*}
    x_0, \ldots, x_{k-1} \mid \op{i}(t_1, \ldots, t_{\arity{i}})
  \end{equation*}
  %
  is interpreted as the composition of maps
  %
  \begin{equation*}
    \xymatrix@+6em{
      {|I|^k} \ar[r]^{(\sem{t_0}_I, \ldots, \sem{t_{\arity{i}}}_I)}
      &
      {|I|^{\arity{i}}} \ar[r]^{\sem{\op{i}}_I}
      &
      {|I|}
    }
  \end{equation*}
  %
  where we elided the contexts $x_0, \ldots, x_{k-1}$ for the sake of brevity.
\end{enumerate}

\begin{example}
  One interpretation of the signature from Example~\ref{ex:monoid-signature} is given by
  the carrier set $\RR$ and the interpretations of operation symbols
  %
  \begin{align*}
    \sem{\mathsf{u}}() &= 1 + \sqrt{5}, \\
    \sem{\mathsf{m}}(a, b) &= a^2 + b^3.
  \end{align*}
  %
  The term in context $x, y \mid \mathsf{m}(\mathsf{u}, \mathsf{m}(x, x))$ is interpreted
  as the map $\RR \to \RR$, given by the rule
  %
  \begin{equation*}
    (a, b) \mapsto (a+1)^3 a^6 + 2 (3 + \sqrt{5}).
  \end{equation*}
  %
  The same term in a context $y, x, z$ is interpreted as the map $\RR \times \RR \to \RR$,
  given by the rule
  %
  \begin{align*}
    (a, b, c) &\mapsto (b+1)^3 b^6 + 2 (3 + \sqrt{5}).
  \end{align*}
  %
  These are not the same map, as they do not even have the same domains! It is irrelevant
  whether this interpretation satisfies the monoid laws because we have not considered any
  equations yet.
\end{example}

The previous examples shows why contexts should not be ignored. In mathematical practice
contexts are often relegated to guesswork for the reader, or are handled implicitly. For
example, in real algebraic geometry the solution set of the equation $x^2 + y^2 = 1$ is
either a unit circle in the plane or an infinitely extending cylinder of unit radius in
the space, depending on whether the context might be $x, y$ or $x, y, z$. Which context is
meant is indicated one way or another by the author of the mathematical text.

\subsection{Models of algebraic theories}
\label{sec:models-algebr-theor}

A \emph{model~$M$} of an algebraic theory~$\theory{T}$ is an interpretation of the signature
$\signature{T}$ which validates all the equations $\equations{T}$. That is, for every
equation
%
\begin{equation*}
  x_1, \ldots, x_k \mid \ell = r
\end{equation*}
%
in~$\equations{T}$, the maps
%
\begin{equation*}
  \sem{x_1, \ldots, x_k \mid \ell}_M : |M|^k \to |M|
  \qquad\text{and}\qquad
  \sem{x_1, \ldots, x_k \mid r}_M : |M|^k \to |M|
\end{equation*}
%
are equal.

\begin{example}
  A model $G$ of $\theory{Group}$, cf.\ Example~\ref{ex:theory-group}, is given by a
  carrier set $|G|$ and maps
  %
  \begin{equation*}
    u : \one \to |G|,\qquad
    m : |G| \times |G| \to |G|,\qquad
    i : |G| \to |G|,
  \end{equation*}
  %
  interpreting the operation symbols $\mathsf{u}$, $\mathsf{m}$, and $\mathsf{i}$,
  respectively, such that the equations~$\equations{Group}$. This amounts precisely to
  $(|G|, u, m, i)$ being a group, except that the unit $u$ is viewed as a map
  $\one \to |G|$ instead of an element of~$|G|$.
\end{example}

\begin{example}
  Every algebraic theory has the \emph{trivial model}, whose carrier is the
  singleton~$\one$, and whose operations are interpreted by the unique maps
  $\one^k \to \one$. All equations are satisfied because any two maps $\one^k \to \one$
  are equal.
\end{example}

The previous example explains why one should \emph{not} require $0 \neq 1$ in a ring, as
that prevents the theory of a ring from being algebraic.

\begin{example}
  The empty set is a model of a theory~$\theory{T}$ if, and only if, every operation symbol
  of $\theory{T}$ has non-zero arity.
\end{example}

\begin{example}
  A model of the theory~$\theory{Set_\bullet}$ of a pointed set, cf.\
  Example~\ref{ex:pointed-set}, is a set $S$ together with an element $s \in S$ which
  interprets the constant~$\bullet$.
\end{example}

\begin{example}
  A model of the theory~$\theory{Empty}$, cf.\ Example~\ref{ex:theory-empty}, is the same
  thing as a set.
\end{example}

\begin{example}
  A model of the theory~$\theory{Singleton}$, cf.\ Example~\ref{ex:theory-singleton}, is
  any set with precisely one element.
\end{example}

Suppose $L$ and $M$ are models of a theory~$\theory{T}$. Then we may form the
\emph{product of models} $L \times M$ by taking the cartesian product as the carrier,
%
\begin{equation*}
  |L \times M| = |L| \times |M|,
\end{equation*}
%
and pointwise operations,
%
\begin{equation*}
  \sem{\op{i}}_{M \times L}(a, b) = (\sem{\op{i}}_M(a), \sem{\op{i}}_L(a)).
\end{equation*}
%
The equations $\equations{T}$ are valid in $L \times M$ because they are valid on each
coordinate separately. This construction can be extended to a product of any number of
models, including an infinite one.

\begin{example}
  We may now prove that the theory of a field from Example~\ref{ex:field} is not
  equivalent to an algebraic theory. There are fields of size 2 and 3, namely $\ZZ_2$ and
  $\ZZ_3$. If there were an algebraic theory of a field, then $\ZZ_2 \times \ZZ_3$ would
  be a field too, but it is not, and in fact there is no field of size~6.
\end{example}

\begin{example}
  Similarly, the theory of a finitely generated group from
  Example~\ref{ex:finitely-generated-group} cannot be formulated as an algebraic theory,
  because an infinite product of non-trivial finitely generated groups is not finitely
  generated.
\end{example}

\begin{example}
  Let us give a model of the theory of a $\Cinfty$-ring from
  Example~\ref{ex:Cinfty-theory}. Pick a smooth manifold~$M$, and let the carrier be the
  set $\Cinfty(M, \RR)$ of all smooth scalar fields on~$M$. Given
  $f \in \Cinfty(\RR^n, \RR)$, interpret the operation $\op{f}$ as composition with~$f$,
  %
  \begin{align*}
    \sem{\op{f}} &: \Cinfty(M, \RR)^n \to \Cinfty(M, \RR) \\
    \sem{\op{f}} &: (u_1, \ldots, u_n) \mapsto f \circ (u_1, \ldots, u_n).
  \end{align*}
  %
  We leave it as an exercise to verify that all equations are validated by this
  interpretation.
\end{example}

\subsection{Homomorphisms and the category of models}
\label{sec:homom-categ-models}

Suppose $L$ and $M$ are models of a theory~$\theory{T}$. A
\emph{$\theory{T}$-homomorphism} from $L$ to $M$ is a map $\phi : |L| \to |M|$ between the
carriers which commutes with operations: for every operation symbol $\op{i}$
of~$\theory{T}$, we have
%
\begin{equation*}
  \phi \circ \sem{\op{i}}_L = \sem{\op{i}}_M \circ \underbrace{(\phi, \ldots, \phi)}_{\arity{i}}.
\end{equation*}

\begin{example}
  A homomorphism between groups $G$ and $H$ is a map $\phi : |G| \to |H|$ between the
  carriers such that, for all $a, b \in |G|$,
  %
  \begin{align*}
    \phi(\sem{\mathsf{u}()}_G) &= \sem{\mathsf{u}()}_H,\\
    \phi(\sem{\mathsf{m}}_G(a,b)) &= \sem{\mathsf{m}}_H(\phi(a), \phi(b)),\\
    \phi(\sem{\mathsf{i}}_H(a)) &= \sem{\mathsf{i}}_H(\phi(a)).
  \end{align*}
  %
  This is a convoluted way of saying that the unit maps to the unit, and that $\phi$
  commutes with the group operation and the inverses. Textbooks usually require only that
  a group homomorphism commute with the group operation, which then implies that it also
  preserves the unit and commutes with the inverse.
\end{example}

We may organize the models of an algebraic theory~$\theory{T}$ into a category $\Mod{T}$
whose objects are the models of the theory, and whose morphisms are homomorphisms of the
theory.

\begin{example}
  The category of models of theory $\theory{Group}$, cf.\ Example~\ref{ex:theory-group},
  is the usual category of groups and group homomorphisms.
\end{example}

\begin{example}
  The category of models of the theory $\theory{Set_\bullet}$, cf.\
  Example~\ref{ex:pointed-set}, has as its objects the pointed sets, which are pairs
  $(S, s)$ with $S$ a set and $s \in S$ its \emph{point}, and as homomorphisms
  the point-preserving functions between sets.
\end{example}

\begin{example}
  The category of models of the empty theory $\theory{Empty}$, cf.\
  Example~\ref{ex:theory-empty}, is just the category $\category{Set}$ of sets and
  functions.
\end{example}

\begin{example}
  The category of models of the theory of a singleton $\theory{Singleton}$, cf.\
  Example~\ref{ex:theory-singleton}, is the category whose objects are all the singleton
  sets. There is precisely one morphisms between any two of them. This category is
  equivalent to the trivial category which has just one object and one morphism.
\end{example}

\subsection{Models in a category}
\label{sec:models-category}

So far we have taken the models of an algebraic theory to be sets. More generally, we may
consider models in any category $\category{C}$ with finite products. Indeed, the
definitions of an interpretation and a model from Sections~\ref{sec:interp-of-sign}
and~\ref{sec:models-algebr-theor} may be directly transcribed so that they apply
to~$\category{C}$. An interpretation $I$ in $\category{C}$ is given by
%
\begin{enumerate}
\item an object $|I|$ in $\category{C}$, called the \emph{carrier},
\item for each operation symbol $\op{i}$ a morphism in $\category{C}$
  %
  \begin{equation*}
    \sem{\op{i}}_I : \underbrace{|I| \times \cdots \times |I|}_{\arity{i}} \to |I|.
  \end{equation*}
\end{enumerate}
%
Once again, we abbreviate the $k$-fold product of $|I|$ as $|I|^k$. Notice that a nullary
symbol is interpreted as a morphism $|I|^0 \to |I|$, which is a morphisms from the
terminal object $\one \to |I|$ in~$\category{C}$.

An interpretation~$I$ is extended to $\Sigma$-terms in contexts as follows:
%
\begin{enumerate}
\item the variable $x_1, \ldots, x_k \mid x_i$ is interpreted as the $i$-th projection,
  %
  \begin{equation*}
    \sem{x_0, \ldots, x_{k-1} \mid  x_i}_I = \pi_i : |I|^k \to |I|,
  \end{equation*}
\item a compound term in context
  %
  \begin{equation*}
    x_0, \ldots, x_{k-1} \mid \op{i}(t_1, \ldots, t_{\arity{i}})
  \end{equation*}
  %
  is interpreted as the composition of morphisms
  %
  \begin{equation*}
    \xymatrix@+6em{
      {|I|^k} \ar[r]^{(\sem{t_0}_I, \ldots, \sem{t_{\arity{i}}}_I)}
      &
      {|I|^{\arity{i}}} \ar[r]^{\sem{\op{i}}_I}
      &
      {|I|}
    }
  \end{equation*}
\end{enumerate}
%
A model of an algebric theory~$\theory{T}$ in~$\category{C}$ is an interpretation~$M$ of
its signature $\signature{T}$ which validates all the equations. That is, for every
equation
%
\begin{equation*}
  x_1, \ldots, x_k \mid \ell = r
\end{equation*}
%
in $\equations{T}$, the morphisms
%
\begin{equation*}
  \sem{x_1, \ldots, x_k \mid \ell}_M : |M|^k \to |M|
  \qquad\text{and}\qquad
  \sem{x_1, \ldots, x_k \mid r}_M : |M|^k \to |M|
\end{equation*}
%
are equal.

The definition of a homomorphism carries over to the general setting as well. A
\emph{$\theory{T}$-homomorphism} between $\theory{T}$-models $L$ and $M$ in a category
$\category{C}$ is a morphism $\phi : |L| \to |M|$ in~$\category{C}$ such that, for every
operation symbol~$\op{i}$ in~$\theory{T}$, $\phi$ commutes with the interpretation of
$\op{i}$,
%
\begin{equation*}
  \phi \circ \sem{\op{i}}_L = \sem{\op{i}}_M \circ \underbrace{(\phi, \ldots, \phi)}_{\arity{i}}.
\end{equation*}
%
The $\theory{T}$-models and $\theory{T}$-homomorphisms in a category $\category{C}$ form a
category $\ModC{C}{T}$.

\begin{example}
  A model of the theory of a group $\theory{Group}$ in the category $\category{Top}$ of
  topological spaces and continuous maps is a topological group.
\end{example}

\begin{example}
  What is a model of the theory of a group in the category of groups $\category{Grp}$? Its
  carrier is a group $(G, u, m, i)$ together with group homomorphisms
  $\upsilon : \one \to G$, $\mu : G \times G \to G$, and $\iota : G \to G$ which satisfy
  the group laws. Because $\upsilon$ is a group homomorphism, it maps the unit of the
  trivial group~$\one$ to $u$, so the units $u$ and $\upsilon$ agree. The operations $m$
  and $\mu$ agree too, because
  %
  \begin{equation*}
    \mu(x, y) =
    \mu(m(x, u), m(u, y)) =
    m(\mu(x, u), \mu(u, y)) =
    m(x, y),
  \end{equation*}
  %
  where in the middle step we used the fact that $\mu$ is a group homomorphism. It is now
  clear that the inverses $i$ and $\iota$ agree as well. Furthermore, taking into account
  that $m$ and $\mu$ agree, we also obtain
  %
  \begin{equation*}
    m(x, y) =
    m(m(u, x), m(y, u)) =
    m(m(u, y), m(x, u)) =
    m(y, x).
  \end{equation*}
  %
  The conclusion is that a group in the category of groups is an abelian group. The
  category $\ModC{Grp}{Group}$ is therefore equivalent to the category of abelian groups.
\end{example}

\begin{example}
  A model of the theory of a pointed set, cf.\ Example~\ref{ex:pointed-set}, in the
  category of groups $\category{Grp}$ is a group $(G, u, m, i)$ together with a
  homomorphism $\one \to G$ from the trivial group~$\one$ to $G$. However, there is
  precisely one such homomorphism which therefore need not be mentioned at all. Thus a
  pointed set in groups amounts to a group.
\end{example}


\subsection{Free models}
\label{sec:free-models}

Of special interest are the free models of an algebraic theory. Given an algebraic
theory~$\theory{T}$ and a set $X$, the \emph{free model generated by~$X$} is a model $M$
together with a map $\eta : X \to |M|$ such that, for every $\theory{T}$-model $L$ and
every map $f : X \to |L|$ there is a unique $\theory{T}$-homomorphism
$\overline{f} : M \to L$ for which the following diagram commutes:
%
\begin{equation*}
  \xymatrix{
    {X}
    \ar[r]^{\eta}
    \ar[rd]_{f}
    &
    {|M|}
    \ar[d]^{\overline{f}}
    \\
    &
    {|L|}
  }
\end{equation*}
%
The definition is a bit of a mouthful, but it can be understood as follows: the free model
generated by $X$ is the ``most economical'' way of making a model out of the set~$X$.

\begin{example}
  Let $\finpow{X}$ be the set of all finite subsets of a set~$X$. We show that
  $(\finpow{X}, \emptyset, {\cup})$ is the free semilattice generated by~$X$, cf.\
  Example~\ref{ex:semi-lattice}. The map $\eta : X \to \finpow{X}$ takes $x \in X$ to the
  singleton set $\eta(x) = \set{x}$. Given any semilattice $(L, \bot, {\vee})$ and a map
  $f : X \to |L|$, define the homomorphism $\overline{f} : \finpow{X} \to |L|$ by
  %
  \begin{equation*}
    \overline{f}(\set{x_1, \ldots, x_n}) = f(x_1) \wedge \cdots \wedge f(x_n).
  \end{equation*}
  %
  Clearly, the required diagram commutes because
  %
  \begin{equation*}
    \overline{f}(\eta(x)) = \overline{f}(\set{x}) = f(x).
  \end{equation*}
  %
  If $g : \finpow{X} \to |L|$ is another homomorphism satisfying $g \circ \eta = f$ then
  %
  \begin{multline*}
    g(\set{x_1, \ldots, x_n})
    = g(\eta(x_1) \cup \cdots \cup \eta(x_n))
    = g(\eta(x_1)) \wedge \cdots \wedge g(\eta(x_n)) \\
    = f(x_1) \wedge \cdots \wedge f(x_n)
    = \overline{f}(\set{x_1, \ldots, x_n}),
  \end{multline*}
  %
  hence $\overline{f}$ is indeed unique.
\end{example}

\begin{example}
  The free model generated by~$X$ of the theory of a pointed set, cf.\
  Example~\ref{ex:pointed-set}, is the disjoint union $X + \one$ whose elements are of the
  form $\iota_0(x)$ for $x \in X$ and $\iota_1(y)$ for $y \in \one$. The point is the
  element $\iota_1(\star)$. The map $\eta : X \to X + \one$ is the canonical
  inclusion~$\iota_0$.
\end{example}

\begin{example}
  The free model generated by~$X$ of the empty theory, cf.\ Example~\ref{ex:theory-empty},
  is~$X$ itself, with $\eta : X \to X$ the identity map.
\end{example}

\begin{example}
  The free model generated by~$X$ of the theory of a singleton, cf.\
  Example~\ref{ex:theory-singleton}, is the singleton set~$\one$, with $\eta : X \to \one$
  the only map it could be. This example shows that~$\eta$ need not be injective.
\end{example}

Every alebraic theory~$\theory{T}$ has free models. Let us sketch the construction of the
free $\theory{T}$-model generated by a set~$X$.

Given a signature $\Sigma$ and a set~$X$, define~$\Tree{\Sigma}{X}$ to be the set of
well-founded trees built inductively as follows:
%
\begin{enumerate}
\item for each $x \in X$, there is a tree $\leaf{x} \in \Tree{\Sigma}{X}$,
\item for each operation symbol $\op{i}$ and trees
  $t_1, \ldots, t_{\arity{i}} \in \Tree{\Sigma}{X}$, there is a tree
  $\op{i}(t_1, \ldots, t_n) \in \Tree{\Sigma}{X}$.
\end{enumerate}
%
Notice that the $\Sigma$-terms in context $x_1, \ldots, x_n$ are precisely the trees
in $\Tree{\Sigma}{\set{x_1, \ldots, x_n}}$.

Suppose $x_1, \ldots, x_n \mid t$ is a $\Sigma$-term in context, and we are given an
assignment $\sigma : \set{x_1, \ldots, x_n} \to \Tree{\Sigma}{X}$ of trees to
variables in the context. Then we may build the tree $\sigma(t)$ inductively as follows:
%
\begin{enumerate}
\item $\sigma(t) = \sigma(x_i)$ if $t = x_i$,
\item $\sigma(t) = \op{i}(\sigma(t_1), \ldots, \sigma(t_n))$ if
  $t = \op{i}(t_1, \ldots, t_n)$.
\end{enumerate}
%
In words, the tree $\sigma(t)$ is obtained by replacing each variable~$x_i$ in~$t$ with
the corresponding tree $\sigma(x_i)$.


Given a theory~$\theory{T}$, let $\approx_\theory{T}$ be the least equivalence relation on
$\Tree{\signature{T}}{X}$ such that:
%
\begin{enumerate}
\item for every equation $x_1, \ldots, x_n \mid \ell = r$ in $\equations{T}$ and for every
  assignment $\sigma : \set{x_1, \ldots, x_n} \to \Tree{\signature{T}}{X}$, we have
  %
  \begin{equation*}
    \sigma(\ell) \approx_{\theory{T}} \sigma(r).
  \end{equation*}
  %
\item $\approx_{\theory{T}}$ is a $\signature{T}$-congruence: for every operation symbol
  $\op{i}$ in $\signature{T}$, if
  %
  \begin{equation*}
    s_1 \approx_{\theory{T}} t_1,
    \quad \ldots \quad,
    s_{\arity{i}} \approx_{\theory{T}} t_{\arity{i}}
  \end{equation*}
  %
  then
  %
  \begin{equation*}
    \op{i}(s_1, \ldots, s_{\arity{i}}) \approx_{\theory{T}}
    \op{i}(t_1, \ldots, t_{\arity{i}}).
  \end{equation*}
\end{enumerate}
%
Define the carrier of the free model $\Free{T}{X}$ to be the quotient set
%
\begin{equation*}
  |\Free{T}{X}| = \Tree{\signature{T}}{X} / {\approx_{\theory{T}}}.
\end{equation*}
%
Let $[t]$ be the $\approx_{\theory{T}}$-equivalence class of
$t \in \Tree{\signature{T}}{X}$. The intepretation of the operation symbol $\op{i}$ in
  $\Free{T}{X}$ is the map $\sem{\op{i}}_{\Free{T}{X}}$ defined by
%
\begin{equation*}
  \sem{\op{i}}_{\Free{T}{X}}([t_1], \ldots, [t_{\arity{i}}]) =
  [\op{i}(t_1, \ldots, t_{\arity{i}})].
\end{equation*}
%
The map $\eta_X : X \to \Free{T}{X}$ is defined by
%
\begin{equation*}
  \eta_X(x) = [\leaf{x}].
\end{equation*}
%
To see that we successfully defined a $\theory{T}$-model, and that it is freely generated
by~$X$, one has to verify a number of mostly straightfoward technical details, which we
omit.

When a theory $\theory{T}$ has no equations the free models generated by~$X$ is just the
set of trees $\Tree{T}{X}$ because the relation $\approx_{\theory{T}}$ is equality.

\subsection{Operations with general arities and parameters}
\label{sec:oper-gener-arit-param}

We have so far presented the classic topic of algebraic theories, as studied in
universal algebra. To get a better fit with computational effects, we need to
generalize to generalize the notion of operation.

\subsubsection{General arities}
\label{sec:general-arities}

We shall require operations that accept an arbitrary, but fixed collection of
arguments. One might expect that the correct way to do so is to allow arities to
be ordinal or cardinal numbers, as these generalize natural numbers, but that
would be a thoroughly non-computational idea. Instead, let us observe that an
$n$-ary cartesian product
%
\begin{equation*}
  \underbrace{X \times \cdots \times X}_{n}
\end{equation*}
%
is isomorphic to the exponential $X^{[n]}$, where
$[n] = \set{0, 1, \ldots, n-1}$. Recall that an exponential $B^A$ is the set of
all functions $A \to B$, and in fact we shall use the notations $B^A$ and
$A \to B$ interchangably. If we replace $[n]$ by an arbitrary set~$A$, then we
can think of a map
%
\begin{equation*}
  X^A \to X
\end{equation*}
%
as taking $A$-many arguments. However, rather than trying to devise a notation
for tuples with $A$-many components, we shall use what is already provided to us
by the exponentials, namely the $\lambda$-calculus.

\begin{example}
  Let us rewrite the group operations in the new notation. The empty set
  $\emptyset$, the singleton $\one = \set{\star}$, and the set of boolean values
  %
  \begin{equation*}
    \bool = \set{\false, \true}
  \end{equation*}
  %
  will serve as arities. We shall use the conditional statement
  %
  \begin{equation*}
    \cond{b}{x}{y}
  \end{equation*}
  %
  as a synonym for what is usually written as definition by cases,
  %
  \begin{equation*}
  \begin{cases}
      x & \text{if $b = \true$,}\\
      y & \text{if $b = \false$.}
    \end{cases}
  \end{equation*}
  %
  Now a group is given by a carrier set $G$ together with maps
  %
  \begin{align*}
    \mathsf{u} &: G^\emptyset \to G,\\
    \mathsf{m} &: G^\bool \to G,\\
    \mathsf{i} &: G^\one \to G,
  \end{align*}
  %
  satisfying the usual group laws, which we ought to write down using the
  $\lambda$-notation. The associativity law is written like this:
  %
  \begin{multline*}
    \mathsf{m}(\lam{b} \cond{b}{\mathsf{m}(\lam{c}\cond{c}{x}{y})}{z}) = \\
    \mathsf{m}(\lam{b} \cond{b}{x}{\mathsf{m}(\lam{c} \cond{c}{y}{z})}).
  \end{multline*}
  %
  Here is the right inverse law, where $\mathsf{O}_X : \emptyset \to X$ is
  the unique map from $\emptyset$ to~$X$:
  %
  \begin{equation*}
    \mathsf{m}(\lam{b} \cond{b}{x}{\mathsf{i}(\lam{\_}{x})}) =
    \mathsf{u}(\mathsf{O}_G).
  \end{equation*}
  %
  The symbol $\_$ indicates that the argument of the $\lambda$-abstraction is
  ignored, i.e., that the function defined by the abstraction is constant. One
  more example might help: $x \cdot x$ may be written not only as
  $\mathsf{m}(\lam{b} \cond{b}{x}{x})$ but also as $\mathsf{m}(\lam{\_} x)$.
\end{example}

Such notation is not appropriate for performing algebraic manipulations, but it
turns out to be quite convenient for programming with algebraic effects.


\subsubsection{Operations with parameters}
\label{sec:oper-with-param}

To motivate our second generalization, consider the theory of a module~$V$ over
a ring~$R$ (if you are not familiar with modules, think of the elements of $V$
as vectors and the elements of~$R$ as scalars). For it to be an algebraic
theory, we need to deal with scalar multiplication ${\cdot} : R \times M \to M$,
because it does not fit the established pattern. There are three possibilities:
%
\begin{enumerate}
\item We could introduce \emph{multi-sorted} alebraic theories whose operations
  take arguments from several carrier sets.
\item Instead of having a single binary operation taking a scalar and a vector,
  we could have many unary operations taking a vector, one for each scalar.
\item We could view the scalar as an additional \emph{parameter} of a
  unary opertion on vectors.
\end{enumerate}
%
The second and the third options are quite similar, but they do differ in their
treatment of parameters. In one case the parameters are part of the indexing of
the signature, while in the other they are properly part of the algebraic
theory. We shall adopt operations with parameters because it best reflects
computational effects that arise in practice.

\subsection{Algebraic theories with parametrized operations and general arities}
\label{sec:algebr-theor-with}

Let us restate the definitions of signatures and algebraic operations, with all
the generalizations taken into account. For simplicity we work with sets and
functions, and leave generalizations to other categories for another occasion.

A \emph{signature $\Sigma$} is given by a collection
$\family{(\op{i}, P_i, A_i)}{i}$ of operation symbols $\op{i}$ with associated
\emph{parameter sets~$P_i$} and \emph{arities~$A_i$}. The symbols may be
anything, although we think of them as syntactic entities, while $P_i$'s and
$A_i$'s are sets.

The \emph{well-founded trees $\Tree{\Sigma}{X}$} over~$\Sigma$ generated by a
set~$X$ form a set defined inductively as follows:
%
\begin{enumerate}
\item $\leaf{x} \in \Tree{\Sigma}{X}$ for every generator $x \in X$,
\item if $p \in P_i$ and $\kappa : A_i \to \Tree{\Sigma}{X}$ then
  $\op{i}(p, \kappa) \in \Tree{\Sigma}{X}$.
\end{enumerate}
%
We equate \emph{$\Sigma$-terms in context $x_1, \ldots, x_k$} with the trees
$\Tree{\Sigma}{\set{x_1, \ldots, x_k}}$.

An \emph{interpretation $I$ of a signature $\Sigma$} is given by:
%
\begin{enumerate}
\item a carrier set $|I|$,
\item for each operation symbol $\op{i}$ with parameter set~$P_i$ and arity~$A_i$,
  a map
  %
  \begin{equation*}
    \sem{\op{i}}_I : P_i \times |I|^{A_i} \longrightarrow |I|.
  \end{equation*}
\end{enumerate}
%
The interpretation $I$ may be extended to terms in contexts. A term
$x_1, \ldots, x_k \mid t$ is interpreted as a map
%
\begin{equation*}
  \sem{x_1, \ldots, x_k \mid t}_I : |I|^k \to |I|
\end{equation*}
%
as follows:
%
\begin{enumerate}
\item the variable $x_i$ is interpreted as the $i$-th projection,
  %
  \begin{equation*}
    \sem{x_0, \ldots, x_{k-1} \mid  x_i}_I = \pi_i : |I|^k \to |I|,
  \end{equation*}
\item a compound term in context
  %
  \begin{equation*}
    x_1, \ldots, x_k \mid \op{i}(p, \kappa)
  \end{equation*}
  %
  is interpreted as the map
  %
  \begin{align*}
    \sem{\op{i}(p, \kappa)}_I &: |I|^k \longrightarrow |I| \\
    \sem{\op{i}(p, \kappa)}_I &:
      \eta \mapsto
      \sem{\op{i}}_I(p, \lam{a \in A_i} \sem{\kappa(a)}_I(\eta)),
  \end{align*}
\end{enumerate}

A \emph{$\Sigma$-equation} is a pair of $\Sigma$-terms $\ell$ and $r$ in context
$x_1, \ldots, x_k$, written
%
\begin{equation*}
  x_1, \ldots, x_k \mid \ell = r.
\end{equation*}
%
Given an interpretation $I$ of signature $\Sigma$, we say that such an equation
is \emph{valid} for~$I$ when the interpretations of $\ell$ and $r$ give the same
map.

An \emph{algebraic theory $\theory{T} = (\signature{T}, \equations{T})$} is
given by a signature $\signature{T}$ and a collection of $\Sigma$-equations
$\equations{T}$. A \emph{$\theory{T}$-model} is an interpretation for
$\signature{T}$ which validates all the equations~$\equations{T}$.

The notions of $\theory{T}$-morphisms and the category $\Mod{}{T}$ of
$\theory{T}$-models and $\theory{T}$-morphisms may be similarly generalized. We
do not repeat the definitions here, as they are almost the same. You should
convince yourself that every algebraic theory has a free model, which is still
built as a quotient of the set of well-founded trees.

\section{Computational effects as algebraic operations}
\label{sec:comp-effects-as}

\section{Algebraic handlers}
\label{sec:algebraic-handlers}



\end{document}
