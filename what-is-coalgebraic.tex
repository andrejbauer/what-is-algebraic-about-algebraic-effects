\documentclass{amsart}

\usepackage[T1]{fontenc}
\usepackage[utf8]{inputenc}
\usepackage[hidelinks]{hyperref}
\usepackage{amsmath,amssymb,amsthm}
\usepackage{times}
\usepackage{xypic}

\newcommand{\NN}{\mathbb{N}} % natural numbers
\newcommand{\RR}{\mathbb{R}} % real numbers
\newcommand{\ZZ}{\mathbb{Z}} % integers

\newcommand{\theory}[1]{\mathsf{#1}} % A named theory
\newcommand{\signature}[1]{\Sigma_{\theory{#1}}} % The signature of a theory
\newcommand{\equations}[1]{\mathcal{E}_{\theory{#1}}} % Equations of a theory

\newcommand{\Mod}[1]{\text{\textbf{Mod}}(\theory{#1})} % Models of a theory in Set
\newcommand{\ModC}[2]{\text{\textbf{Mod}}_{\category{#1}}(\theory{#2})} % Models of a theory in a category
\newcommand{\ComodC}[2]{\text{\textbf{CoMod}}_{\category{#1}}(\theory{#2})} % Comodels of a theory in a category
\newcommand{\opcat}[1]{\ensuremath{#1^{\text{\textbf{op}}}}}
\newcommand{\from}{\leftarrow}

\newcommand{\category}[1]{\text{\textbf{#1}}} % A named category
\newcommand{\Set}{\category{Set}} % The category of sets

\newcommand{\Free}[2]{F_{\theory{#1}}(#2)} % free model

\newcommand{\all}[1]{\forall #1 \,.\,} % universal quantifier
\newcommand{\some}[1]{\exists #1 \,.\,} % existential quantifier
\newcommand{\set}[1]{\{#1\}} % set description
\newcommand{\such}{\mid}

\newcommand{\lam}[1]{\lambda #1 \,.\,}

\newcommand{\family}[2]{\{#1\}_{#2}} % a family

\newcommand{\finpow}[1]{\mathcal{P}_{{<}\omega}(#1)} % finite powerset

\newcommand{\Tree}[2]{\mathsf{Tree}_{#1}(#2)} % trees over a signature
\newcommand{\leaf}[1]{\mathsf{leaf}(#1)} % the embedding of generators into trees

\newcommand{\op}[1]{\mathsf{op}_{#1}} % an operation symbol
\newcommand{\arity}[1]{\mathsf{ar}_{#1}} % arity of a symbol

\newcommand{\one}{\mathsf{1}} % the terminal object

\newcommand{\Cinfty}{\mathcal{C}^\infty}

\newcommand{\sem}[1]{[\![#1]\!]} % semantic bracket

\newcommand{\bool}{\mathsf{bool}} % booleans
\newcommand{\true}{\mathsf{true}}
\newcommand{\false}{\mathsf{false}}
\newcommand{\cond}[3]{\mathsf{if}\;#1\;\mathsf{then}\;#2\;\mathsf{else}\;#3}
\newcommand{\conf}[2]{\langle #1, #2 \rangle}

%%% Macros for the programming language

% General syntactic constructs
\newcommand{\kode}[1]{\mathtt{#1}}

% Core syntax
\newcommand{\seq}[2]{\kode{do}\; #1 \leftarrow #2 \;\kode{in}\;}
\newcommand{\conditional}[3]{\kode{if}\; #1 \;\kode{then}\; #2 \;\kode{else}\; #3}
\newcommand{\fun}[1]{\kode{fun}\; #1 \mapsto}
\newcommand{\handler}{\kode{handler}\;}
\newcommand{\opgen}[1]{\op{#1}}
\newcommand{\opcall}[4]{#1(#2; #3.\,#4)}
\newcommand{\opclause}[3]{#1(#2; #3) \mapsto}
\newcommand{\return}[1]{\kode{return}\;#1}
\newcommand{\retclause}[1]{\return{#1} \mapsto}
\newcommand{\withhandle}[2]{\kode{with}\; #1\; \kode{handle}\; #2}



{\theoremstyle{definition}
\newtheorem{definition}{Definition}[section]
\newtheorem{example}[definition]{Example}
}

\begin{document}

\title{What is coalgebraic about algebraic effects and handlers?}
\thanks{The author acknowledges that this material is based upon work
 supported by the Air Force Office of Scientific Research under award number
 FA9550-17-1-0326.}
\author{Matija Pretnar}

\maketitle

As the name suggests and as Andrej Bauer explained in his note \emph{``What is algebraic about algebraic effects and handlers?''} above, algebraic effects and their handlers have a strong connection with algebras. In this note, I will try to show that they also have some relatively unexplored coalgebraic features. We will continue where the previous note left off, so make sure to read that one first.


\section{Dualizing models of algebraic theories}
\label{sec:dualizing-models}

The main objects of our study will be \emph{comodels} of an algebraic theory. Let us start with an abstract categorical definition and work back from there to a useful characterization. Those not interested in categorical manipulations may skip to Section~\ref{sec:comodels} and take that as a starting point.

Recall that for a given category $\category{C}$, the dual category $\opcat{\category{C}}$ has the same objects as $\category{C}$, whereas its morphisms go in the other direction: morphisms between $A$ and $B$ in $\opcat{\category{C}}$ correspond exactly to morphisms $B \to A$ in $\category{C}$, which is why we will write them as $A \from B$, keeping objects in the same place and reversing the arrows. We use bold letters to write $\text{\textbf{op}}$, the standard notation for a dual category, in order to distinguish it from $\op{}$, the standard notation for an algebraic operation.

Take $\category{C}$ to be any category with finite \emph{coproducts}. We roughly follow Plotkin \& Power~\cite{power} and define the category of $\theory{T}$-comodels $\ComodC{C}{T}$ and $\theory{T}$-cohomomorphisms to be $\opcat{(\ModC{\opcat{C}}{T})}$, i.e. the dual of $\theory{T}$-models in the dual of $\category{C}$. Let us express these in more familiar terms.

First, consider the objects of $\ComodC{C}{T}$. Since a category has the same objects as its dual, these correspond to objects of $\ModC{\opcat{C}}{T}$, which are given by:
%
\begin{enumerate}
\item An object $|W|$ in $\opcat{\category{C}}$, called the \emph{carrier}. Again, a category has the same objects as its dual, these correspond to objects $|W|$ in $\category{C}$.
\item For each operation symbol $\op{i}$ a morphism in $\opcat{\category{C}}$
  %
  \begin{equation*}
    \sem{\op{i}}_W : \underbrace{|W| \times \cdots \times |W|}_{\arity{i}} \from |W|,
  \end{equation*}
  which validates all the equations of the theory (we will return to this in a moment). However, products in $\opcat{\category{C}}$ correspond to coproducts in $\category{C}$ (which is why we required $\category{C}$ to have finite coproducts), thus the above family of morphisms in $\opcat{\category{C}}$ corresponds to the family of morphisms
  %
    \begin{equation*}
    \sem{\op{i}}^W : |W| \to \underbrace{|W| + \cdots + |W|}_{\arity{i}},
  \end{equation*}
  in $\category{C}$, which we call \emph{cooperations}. We abbreviate an $n$-ary coproduct $|W| + \cdots + |W|$ by $n \cdot |W|$.
\end{enumerate}
Since the category $\ComodC{C}{T} = \opcat{(\ModC{\opcat{C}}{T})}$ has the same objects as its dual $\ModC{\opcat{C}}{T}$, such objects are exactly $\theory{T}$-comodels.

A family of cooperations on a model validates the equations of the theory~$\theory{T}$ if for every equation
%
\begin{equation*}
  x_1, \ldots, x_k \mid \ell = r
\end{equation*}
%
in~$\equations{T}$, the maps $\sem{x_1, \ldots, x_k \mid \ell}^W$ and $\sem{x_1, \ldots, x_k \mid r}^W$ are equal. Interpretations of $\Sigma$-terms are, of course, defined dually. A $\Sigma$-term in context
%
\begin{equation*}
  x_0, \ldots, x_{k-1} \mid t
\end{equation*}
%
is interpreted by a map
%
\begin{equation*}
  \sem{x_0, \ldots, x_{k-1} \mid t}^I : |I| \to k \cdot |I|,
\end{equation*}
%
as follows:
%
\begin{enumerate}
\item the variable $x_i$ is interpreted as the $i$-th injection,
  %
  \begin{equation*}
    \sem{x_0, \ldots, x_{k-1} \mid  x_i}^I = \iota_i : |I| \to k \cdot |I|,
  \end{equation*}
\item a compound term in context
  %
  \begin{equation*}
    x_0, \ldots, x_{k-1} \mid \op{i}(t_1, \ldots, t_{\arity{i}})
  \end{equation*}
  %
  is interpreted as the composition of maps
  %
  \begin{equation*}
    \xymatrix@+6em{
      {|I|} \ar[r]^{\sem{\op{i}}^I}
      &
      {{\arity{i}} \cdot |I|} \ar[r]^{[\sem{t_0}_I, \ldots, \sem{t_{\arity{i}}}_I]}
      &
      {k \cdot |I|}
    }
  \end{equation*}
\end{enumerate}
%
where for morphisms $f_i \colon A_i \to B$, we define the morphism $[f_1, \dots, f_k] \colon A_1 + \cdots + A_k \to B$ by $[f_1, \dots, f_k](\iota_j(x)) = f_j(x)$.

Finally, morphisms between comodels $W$ and $V$ in $\ComodC{C}{T}$ are exactly $\theory{T}$-homomorphisms between $V$ and $W$ in its dual $\ModC{\opcat{C}}{T}$. These in turn correspond to morphisms $\phi \colon |V| \from |W|$ in $\opcat{C}$ that commute with operations: for every operation symbol $\op{i}$ of~$\theory{T}$, we have
%
\begin{equation*}
  \phi \circ \sem{\op{i}}_V = \sem{\op{i}}_W \circ \underbrace{(\phi, \ldots, \phi)}_{\arity{i}}
\end{equation*}
in the category $\opcat{C}$. If we state this in terms of $\category{C}$, we get a morphism $\phi \colon |W| \to |V|$ that commutes with cooperations: for every operation symbol $\op{i}$ of~$\theory{T}$, we have
%
\begin{equation*}
  \sem{\op{i}}_V \circ \phi = \underbrace{[\phi, \ldots, \phi]}_{\arity{i}} \circ \sem{\op{i}}_W.
\end{equation*}

Let us now move to the category of sets, and while doing this, let us also switch to operations with parameters and a general arity. We see that in sets, $k \cdot |W| = [k] \times |W|$, so an operation symbol $\op{}$ with parameter set~$P$ and arity~$A$ can be interpreted as a cooperation $P \cdot |W| \to A \cdot |W|$.

\section{Comodels of algebraic theories}
\label{sec:comodels}

A \emph{cointerpretation $I$ of a signature $\Sigma$} is given by:
%
\begin{enumerate}
\item a carrier set $|I|$,
\item for each operation symbol $\op{i}$ with parameter set~$P_i$ and arity~$A_i$,
  a map
  %
  \begin{equation*}
    \sem{\op{i}}^kI : P_i \times |I| \longrightarrow A_i \times |I|,
  \end{equation*}
  which we call a \emph{cooperation}.
\end{enumerate}
%
The interpretation $I$ may be extended to terms in contexts. A term
$x_1, \ldots, x_k \mid t$ is interpreted as a map
%
\begin{equation*}
  \sem{x_1, \ldots, x_k \mid t}_I : |I| \to k \cdot |I|
\end{equation*}
%
as follows (for those that skipped the first section, $k \cdot |I| = \underbrace{|I| + \cdots + |X|}k$):
%
\begin{enumerate}
\item the variable $x_i$ is interpreted as the $i$-th injection,
  %
  \begin{equation*}
    \sem{x_0, \ldots, x_{k-1} \mid  x_i}_I = \iota_i : |I| \to k \cdot |I|,
  \end{equation*}
\item a compound term in context
  %
  \begin{equation*}
    x_1, \ldots, x_k \mid \op{i}(p, \kappa)
  \end{equation*}
  %
  is interpreted as the map
  %
  \begin{align*}
    \sem{\op{i}(p, \kappa)}_I &: |I| \longrightarrow k \cdot |I| \\
    \sem{\op{i}(p, \kappa)}_I &:
      w \mapsto (a, \kappa)
      \sem{\op{i}}_I(p, \lam{a \in A_i} \sem{\kappa(a)}_I(\eta)),
  \end{align*}
\end{enumerate}

A \emph{$\Sigma$-equation} is a pair of $\Sigma$-terms $\ell$ and $r$ in context $x_1, \ldots, x_k$, written
%
\begin{equation*}
  x_1, \ldots, x_k \mid \ell = r.
\end{equation*}
%
Given an cointerpretation $I$ of signature $\Sigma$, we say that such an equation is \emph{valid} for~$I$ when the interpretations of $\ell$ and $r$ give the same map.

An \emph{algebraic theory $\theory{T} = (\signature{T}, \equations{T})$} is
given by a signature $\signature{T}$ and a collection of $\Sigma$-equations
$\equations{T}$. A \emph{$\theory{T}$-comodel} is a cointerpretation for
$\signature{T}$ which validates all the equations~$\equations{T}$.

\begin{example}
Take the signature~$\Sigma$ with a binary operation symbol $\mathsf{choose}$. A cointerpretation of $\Sigma$ is given by a set $S$ together with a map $\sem{\mathsf{choose}}^S \colon S \to 2 \times S$. This can be seen as a boolean stream, where on each call $\sem{\mathsf{choose}}^S$ produces a boolean and the remainder of the stream.
\end{example}

\begin{example}
A state mapping locations from $L$ into values from $V$ can be described by a signature~$\Sigma$ consisting of two operation symbols: $\mathsf{get}$ with parameter set $L$ and arity $V$, and a unary $\mathsf{set}$ with parameter set $L \times V$. The cointerpretation of $\Sigma$ amounts to a set $S$ with maps $\sem{\mathsf{get}} \colon L \times S \to V \times S$ and $\sem{\mathsf{set}} \colon (L \times V) \times S \to S$. One such set would be the set of functions $V^L$, where $\sem{\mathsf{get}}(\ell, s) = s(\ell)$ and $\sem{\mathsf{get}}((\ell, v), s) = s[\ell \mapsto v]$, i.e. a function that behaves as $s$, but maps $\ell$ to $v$.
\end{example}

\begin{example}
Output is represented with a unary operation symbol $\mathsf{print}$ with a parameter set $O$ representing the output alphabet. Cointerpretations of such signature are given by sets $W$ operation A state mapping locations from $L$ into values from $V$ can be described by a signature~$\Sigma$ consisting of two operation symbols: $\mathsf{get}$ with parameter set $L$ and arity $V$, and a unary $\mathsf{set}$ with parameter set $L \times V$. The cointerpretation of $\Sigma$ amounts to a set $S$ with maps $\sem{\mathsf{get}} \colon L \times S \to V \times S$ and $\sem{\mathsf{set}} \colon (L \times V) \times S \to S$. One such set would be the set of functions $V^L$, where $\sem{\mathsf{get}}(\ell, s) = s(\ell)$ and $\sem{\mathsf{get}}((\ell, v), s) = s[\ell \mapsto v]$, i.e. a function that behaves as $s$, but maps $\ell$ to $v$.
\end{example}



\begin{example}
Note that if $\Sigma$ contains a nullary operation symbol $\op : 0$, no nontrivial cointerpretation of $\Sigma$ exists, as that would entail a set $|W|$ and a map $|W| \to \emptyset$, which is possible only if $|W|$ is itself $\emptyset$.
\end{example}

\section{Using comodels for interpreting hardware}

We can think of comodels~$W$ as hardware on which operations are executed. The carrier~$|W|$ represents the state, while cooperations $|W| \times P \to |W| \times A$ take the initial state and a parameter, and produce the result and the altered state.

We can use this to execute programs, which, recall, are represented with models. Take a model~$M$ representing the software and a comodel~$W$ representing the hardware  on which it is run. Next, take an initial configuration $\conf{m}{w}$, where $m \in M$ is a program and $w \in W$ is the inital state. If $m$ is of the form $\return v$, our program has terminated with the resulting value $v$ in a final state~$w$. But if $m$ calls an operation and is of the form $\op{M}(p, \kappa)$, we can execute $\op{}$ on $W$. By applying the cooperation $\sem{\op}^W$ to the parameter $p$ and the world~$w$, we get a result~$a$ and a new world~$w'$. We pass $a$ to $\kappa$ to get a continuation, which we resume in the new state $w'$.

This relation induces an equivalence $\equiv$ on the set of configurations $M \times W$, given by:
\[
  \conf{\op{M}(p, \kappa)}{w} \equiv \conf{\kappa(a)}{w'} \quad\text{where $\op{}^W(p; w) = (a, w')$}
\]
We label the resulting quotient set $(M \times W) / \equiv$ by $M \otimes W$ and call it the \emph{tensor product}, because of its similarity to how scalars in a tensor product pass from one component to the other.

In case the theory~$\theory{T}$ is nontrivial, we must check that the relation is well-defined, for $\op{M}(p, \kappa)$ may be equivalent to some other $m' \in M$. Since $W$ is also a comodel, $\op{}^W$ obeys the same equations. We leave the proof as an exercise for the reader.

\section{Tensoring models and comodels of algebraic theories}
\label{sec:tensoring-models}

We have used comodels to represent the hardware that executes the operations that escape all handlers and reach the toplevel. Another frequent viewpoint is to represent it with a ``default handler'' which sits at the top and encompasses the whole executed computation. Let us compare the two representations and see why the former seems a better fit. First, we can show that any such hardware can indeed be simulated by a paramater-passing handler. For a comodel $W$ with cooperations, we can define a model $(X \times W)^W$ where the program computing values from $V$ can be ...

\end{document}
