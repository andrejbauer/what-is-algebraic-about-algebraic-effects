\documentclass{article}
\usepackage{mathtools}
\usepackage{tikz}
\newcommand{\todo}[1]{\textcolor{red}{#1}}
\newcommand{\cond}[1]{\quad(#1)}

% Meta-syntax
\newcommand{\bnfis}{\mathrel{\;{:}{:}\!=}\;}
\newcommand{\bnfor}{\mathrel{\;\big|\;}}
\newcommand{\defeq}{\mathrel{\;\stackrel{\text{def}}{=}\;}}

% General syntactic constructs
\newcommand{\kord}[1]{\mathbf{#1}}
\newcommand{\kop}[1]{\;\mathbf{#1}\;}
\newcommand{\kpre}[1]{\mathbf{#1}\;}

\newcommand{\const}[1][k]{\mathsf{#1}}
\newcommand{\type}[1]{\mathsf{#1}}
\newcommand{\op}[1][op]{\mathsf{#1}}

% Core syntax
\newcommand{\seq}[2]{\kpre{do} #1 \leftarrow #2 \kop{in}}
\newcommand{\conditional}[3]{\kpre{if} #1 \kop{then} #2 \kop{else} #3}
\newcommand{\fls}{\kord{false}}
\newcommand{\fun}[1]{\kpre{fun} #1 \mapsto}
\newcommand{\recfun}[2]{\kpre{rec} \kpre{fun} #1 \, #2 \mapsto}
\newcommand{\handler}{\kpre{handler}}
\newcommand{\opgen}[1][\op]{#1}
\newcommand{\opcall}[4][\op]{#1(#2; #3.\,#4)}
\newcommand{\opclause}[3][\op]{#1(#2; #3) \mapsto}
\newcommand{\return}{\kpre{return}}
\newcommand{\tru}{\kord{true}}
\newcommand{\retclause}[1]{\return #1 \mapsto}
\newcommand{\withhandle}[2]{\kpre{with} #1 \kop{handle} #2}

% Additional syntax
\newcommand{\anon}{\mathord{\rule[-0.3pt]{1.2ex}{1pt}}}
\newcommand{\decide}{\op[decide]}
\newcommand{\kst}[1]{\mathord{\mathsf{#1}}}
\newcommand{\letin}[1]{\kpre{let} #1 \kop{in}}
\newcommand{\fail}{\op[fail]}
\newcommand{\print}{\op[print]}
\newcommand{\get}{\op[get]}
\newcommand{\set}{\op[set]}
\renewcommand{\read}{\op[read]}
\newcommand{\str}[1]{\mathord{\text{``#1''}}}
\newcommand{\unt}{\kord{()}}

% Operational semantics
\newcommand{\step}{\leadsto}

% Types
\newcommand{\boolty}{\type{bool}}
\newcommand{\C}{\underline{C}}
\newcommand{\D}{\underline{D}}
\newcommand{\Dirt}{\Delta}
\newcommand{\hto}{\Rightarrow}
\newcommand{\opto}{\to}
\newcommand{\E}{\mathop{!}}

% Typing judgements
\newcommand{\ctx}{\Gamma}
\newcommand{\ent}{\vdash}
\newcommand{\sig}{\Sigma}
\newcommand{\T}{\mathrel{:}}

% Denotational semantics
\newcommand{\itp}[1]{\llbracket #1 \rrbracket}
\newcommand{\env}{\mathcal{E}}
\newcommand{\F}[2][\Dirt]{F_{#1}(#2)}


\def\lastname{Pretnar}
\begin{document}

We now informally describe the behaviour of handlers through examples. You may
also prefer to first take a look at the operational semantics given in
Section~\ref{sec:operational-semantics}.

\subsection{Input \& output}

Let us start with input \& output as it is a very simple algebraic effect, but
one which exposes almost all important aspects of handlers. It can be described
by two operations: $\print$, which takes a message to be printed and yields
the unit value~$\unt$, and $\read$, which takes a unit value and yields a
string that was read. For example, a computation that asks the user for his
forename and surname and prints out his full name, is written as:
%
\begin{align*}
  \kst{printFullName} \defeq {}
    & \print \, \str{What is your forename?}; \\
    & \seq{forename}{\read \, \unt} \\
    & \print \, \str{What is your surname?}; \\
    & \seq{surname}{\read \, \unt} \\
    & \print \, (\kst{join} \, forename \, surname)
\end{align*}
%
where $\kst{join}$ is a function that takes two strings and joins them with a
space in the middle.

\subsubsection{Constant input}

A simple example of a handler is:
%
\begin{align*}
  \handler \{
    \opclause[\read]{\anon}{k} k \, \str{Bob}
  \}
\end{align*}
%
which provides a constant input string~$\str{Bob}$ each time $\read$ is called.
We can, of course, generalise it to a function that takes a
string~$s$ and returns a handler that feeds it to $\read$:
\[
  \kst{alwaysRead} \defeq \fun{s} \handler \{
    \opclause[\read]{\anon}{k} k \, s
  \}
\]
This handler works as follows: whenever $\read$ is called, we ignore its unit
parameter and capture its continuation in a function $k$ that expects the
resulting string and resumes the evaluation when applied. Next, instead of calling
$\read$, we evaluate the computation in the handling clause: we resume the
continuation $k$, but instead of reading the string from interactive input, we
yield the constant string~$s$. The handler implicitly continues to
handle the continuation, so any $\read$ in the handled computation again yields
$s$. If the handled computation calls any operation other than $\read$, the call
escapes the handler, but the handler again wraps itself around the continuation
so that it may handle any further $\read$ calls. For example, evaluating
\[
  \withhandle{(\kst{alwaysRead} \, \str{Bob})}{\kst{printFullName}}
\]
first prints out $\str{What is your name?}$ as $\print$ is unhandled. Then, $\read$
is handled so $\str{Bob}$ gets bound to $forename$. Similarly, the second $\print$
is unhandled, and in the second $\read$, $\str{Bob}$ gets bound to $surname$ as
well and finally $\str{Bob Bob}$ is printed out.

It is not obvious whether handlers should continue handling operations in the
continuation, or handle just the first call. Experience with exception handlers
offer us no guidance here, because raised exceptions have no continuation, and so the two
choices are equivalent. As it turns out, the first choice, which we are
settling on in this paper, has nicer denotational semantics, is what one usually desires in practice, and is
perhaps also more intuitive because $\withhandle{h}{c}$ suggests that the whole~$c$ should be handled by $h$. The second choice leads to \emph{shallow
handlers}~\cite{DBLP:conf/icfp/KammarLO13}, which are more convenient for certain uses,
and can be considered a more elementary approach as they can express the usual
handlers through recursion.

\subsubsection{Reversed output}

We can use handlers to not only change what is fed to the continuation, but also
to change the way the continuation is used. For example, to reverse the order of printouts,
we use:
%
\begin{align*}
  \kst{reverse} \defeq \handler \{
    \opclause[\print]{s}{k} k \, \unt; \print \, s
  \}
\end{align*}
%
Here, we handle a $\print$ by first calling the continuation, and only after
this is finished, print out $s$. Since the handler wraps itself around $k$, the
same rule applies for the continuation and so all printouts are reversed.
So, if we define
\[
  \kst{abc} \defeq \print \, \str{A}; \print \, \str{B}; \print \, \str{C}
\]
then $\withhandle{\kst{reverse}}{\kst{abc}}$ prints out first
$\str{C}$, then $\str{B}$, and finally $\str{A}$.

\subsubsection{Collecting output}
\label{sec:collect}

A more useful handler is one that collects all printouts into one big string
and returns it together with the final value:
%
\begin{align*}
  \kst{collect} \defeq \handler \{
  & \retclause{x} \return (x, \str{\,}) \\
  & \opclause[\print]{s}{k} \\
  &\quad \seq{(x, acc)}{k \, \unt} \\
  &\quad \return (x, \kst{join} \, s \, acc)
  \}
\end{align*}
%
If the handled computation does not print anything and just returns some
value~$x$, we need to handle it by returning an empty string in addition to $x$.
But if a computation prints some string $s$, we resume the continuation. Since
this is handled in the same way, it returns the accumulated string $acc$ in
addition to the final value~$x$. Now, we only need to join $s$ with $acc$ and
return it together with $x$. If we handle $\kst{abc}$ with $\kst{collect}$, we
get a pair $(\unt, \str{A B C})$, where~$\unt$ is the unit result of the last
$\print$.

We can also nest handlers, and
%
\[
  \withhandle{\kst{collect}}{(\withhandle{\kst{reverse}}{\kst{abc}})}
\]
%
evaluates to $(\unt, \str{C B A})$. The order in which we nest the handlers is
significant as it is the innermost handler that determines how to first handle the
call. If we switch the handlers in the above example, we get
$(\unt, \str{A B C})$ because $\kst{collect}$ handles all $\print$ calls,
and so none reach the $\kst{reverse}$ handler, which then does nothing.

Alternatively, we could implement the same handler using a technique called
\emph{parameter-passing}~\cite{DBLP:journals/corr/PlotkinP13},
where we transform the handled computation into a function that passes around a parameter,
in our case the accumulated string:
%
\begin{align*}
  \kst{collect'} \defeq \handler \{
  &\retclause{x}
    \fun{acc} \return (x, acc) \\
  &\opclause[\print]{s}{k} \\
  &\quad \fun{acc} (k \, \unt) \, (\kst{join} \, acc \, s)
  \}
\end{align*}
%
When a computation returns a value $x$, there will be no further printouts,
so we can return the given accumulator~$acc$ in addition to $x$. But if $\print$
is called, we resume the continuation by yielding it the expected unit result.
Since the continuation is further handled into a function, we need to
pass $k \, \unt$ the new accumulator, which is $acc$ extended with~$s$. To obtain the collected
output of a computation $c$, we apply the resulting function to the empty
accumulator as:
\[
  (\withhandle{\kst{collect'}}{c}) \, \str{\,}
\]
In Section~\ref{sec:reasoning}, we show that $\kst{collect}$ and $\kst{collect'}$
indeed exhibit equivalent behaviour.
Using parameter-passing, we can also implement a converse handler that feeds
words from a given string to the input. 

\subsection{Exceptions}

Exception handlers are, of course, a special instance of handlers. We represent
exceptions with an operation $\op[raise]$ that takes an exception argument
(e.g.~error message) and yields nothing to the continuation (for more details
on how this can be enforced, see Example~\ref{exa:signatures}).

In practice, exception handlers are rarely reused, but an example of a more
general exception handler is:
\[
  \kst{default} \defeq \fun{x} \handler \{
    \opclause[{\op[raise]}]{\anon}{\anon} \return x
  \}
\]
which returns a default value~$x$ in case the handled computation raises an
exception.

\subsection{Non-determinism}

Handlers can be used not only to override existing effectful behaviour, but to
define new one as well. To implement non-determinism, we take a single operation
$\decide$, which takes a unit parameter, and non-deterministically yields a
boolean. Then, a binary choice can be implemented as a function
%
\begin{align*}
  \kst{choose} \defeq {} & \fun{(x, y)} \\
  &\quad \seq{b}{\decide \, \unt} \\
  &\quad \conditional{b}{(\return x)}{(\return y)}
\end{align*}
%
However, unlike $\print$, we assume no intrinsic behaviour for $\decide$, and we
must use handlers to determine whether to return a fixed result, a random result,
an optimal result, or all results. Without an encompassing handler, an
application $\kst{choose} \, (3, 4)$ is stuck when it encounters the $\decide$
call. The simplest handler for $\decide$ is
%
\begin{align*}
  \kst{pickTrue} \defeq \handler \{
  \opclause[\decide]{\anon}{k} k \, \tru
  \}
\end{align*}
%
which makes each $\decide$ yield $\tru$ to the continuation, so $\kst{choose}$
always chooses the left argument. So, if we define
%
\begin{align*}
  \kst{chooseDiff} \defeq {}
    &\seq{x_1}{\kst{choose} \, (15, 30)} \\
    &\seq{x_2}{\kst{choose} \, (5, 10)} \\
    &\return (x_1 - x_2)
\end{align*}
%
then $\withhandle{\kst{pickTrue}}{\kst{chooseDiff}}$ will choose $15$ for $x_1$ and $5$ for $x_2$,
and will thus evaluate to $\return 10$.

\subsubsection{Maximal result}
\label{sec:pickmax}

With handlers, we can also traverse all possible branches to select the maximal
result:
\begin{align*}
  \kst{pickMax} \defeq \handler \{
  &\opclause[\decide]{\anon}{k} \\
  &\quad \seq{x_t}{k \, \tru} \\
  &\quad \seq{x_f}{k \, \fls} \\
  &\quad \return \kst{max} \, (x_t, x_f) 
  \}
\end{align*}
In this case, evaluating $\withhandle{\kst{pickMax}}{\kst{chooseDiff}}$ will
make the choices needed to get the maximal possible difference $25$, even if
this means choosing the smaller argument of $\kst{choose}$ (in particular, we pick $30$ for $x_1$ and $5$ for $x_2$).

If we included lists in our language, we could adapt $\kst{pickMax}$ to a handler~$\kst{pickAll}$ that select all possible results~\cite{DBLP:journals/jlp/BauerP15}. To do so, the return clause would return a singleton list containing the returned value, while the $\decide$ clause would concatenate the lists $x_t$ and $x_f$ that result from yielding both possible results to the handled continuation.

\subsubsection{Backtracking}
\label{sec:backtrack}

To implement backtracking, where we employ non-deterministic search for a given
solution, we add an operation $\fail$ to signify that no solution exists. Then,
for example:
%
\begin{align*}
  &\recfun{\kst{chooseInt}}{(m, n)} \\
  &\quad \conditional{m > n}{\fail \, \unt}{ \\
  &\quad\quad \seq{b}{\decide \, \unt} \\
  &\quad\quad \conditional{b}{(\return m)}{\kst{chooseInt} \, (m + 1, n)}
  }
\end{align*}
%
is a function that non-deterministically chooses an integer in the interval $[m, n]$,
or fails if this interval is empty, while:
%
\begin{align*}
  \kst{pythagorean} \defeq {} &\fun{(m, n)} \\
  &\quad\seq{a}{\kst{chooseInt} \, (m, n - 1)} \\
  &\quad\seq{b}{\kst{chooseInt} \, (a + 1, n)} \\
  &\quad\conditional{\kst{isSquare} \, (a^2 + b^2)}{(\return (a, b, \sqrt{a^2 + b^2}))}{\fail \, ()}
\end{align*}
%
is a function that searches for an integer Pythagorean triple $(a, b, c)$ such that
${m \leq a < b \leq n}$.
We perform backtracking by handling each $\decide$ by first trying to yield
$\tru$, and if this fails, yield $\fls$:
%
\begin{align*}
  \kst{backtrack} \defeq \handler \{
  &\opclause[\decide]{\anon}{k} \\
  &\quad \withhandle{\\
  &\quad\quad \handler \{ \opclause[\fail]{\anon}{\anon} k \, \fls \} \\
  &\quad}{\\
  &\quad\quad k \, \tru} \}
\end{align*}
%
Then, $\withhandle{\kst{backtrack}}{\kst{pythagorean} \, (m, n)}$
finds $(5, 12, 13)$ for ${(m, n) = (4, 15)}$ but fails for $(m, n) = (7, 10)$.
The exact triple found depends on the implementation of the handler. If, instead, we first tried yielding $\fls$, the resulting triple for $(m, n) = (4, 15)$ would be $(9, 12, 15)$. To get a list of all possible triples, we can use the handler~$\kst{pickAll}$ from Section~\ref{sec:pickmax}, but extended with a clause that handles $\fail$ with an empty list.

\subsection{State}

We represent state with operations $\set$ for setting the state contents, and
$\get$ for reading them. For simplicity, we assume a single memory location that
holds an integer. So, $\set$ takes an integer, stores it, and returns a unit
result, while $\get$ takes a unit parameter, reads the stored integer, and
returns it.

We can use handlers to temporarily alter the stored value or to log all updates.
But we can also use them to implement stateful behaviour
even if we do not assume a built-in one. Like in Section~\ref{sec:collect},
we use a parameter-passing handler to pass around the current state:
%
\begin{align*}
  \kst{state} \defeq \handler \{
  &\opclause[\get]{\anon}{k} \fun{s} (k \, s) \, s \\
  &\opclause[\set]{s}{k} \fun{\anon} (k \, \unt) \, s \\
  &\retclause{x} \fun{\anon} \return x \}
\end{align*}
%
We handle $\get$ with a function that takes the current state $s$ and passes it
first as a result of $\get$ to the continuation, and then again as the unchanged
state. Conversely, we handle $\set$ by first yielding the unit result, and then
applying the handled continuation to the new state $s$ as given in the parameter
of $\set$.

The return clause of $\kst{state}$ ignores the final state, but if we want to
inspect it, we can return it together with the final value by changing the
return clause to:
\[
  \retclause{x} \fun{s} \return (s, x)
\]

\subsubsection{Transactions}

In a similar way, we can implement transactional memory, where we commit the
changed state only after the handled computation successfully terminated with a
value, so in case an exception is raised, the memory contents remain unchanged:
%
\begin{align*}
  \kst{transaction} \defeq \handler \{
  &\opclause[\get]{\anon}{k} \fun{s} (k \, s) \, s \\
  &\opclause[\set]{s}{k} \fun{\anon} (k \, \unt) \, s \\
  &\retclause{x} \fun{s} \set \, s; \return x
  \}
\end{align*}

\end{document}
