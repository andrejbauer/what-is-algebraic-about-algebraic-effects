\documentclass[a4paper,UKenglish]{dagrep-v2018}

\usepackage[utf8]{inputenc}
\usepackage{microtype}%if unwanted, comment out or use option "draft"
\bibliographystyle{plain}%the recommnded bibstyle

\begin{document}

\subject{Report from Dagstuhl Seminar 18172}

\title{Algebraic effects and handlers go mainstream}

\author[1]{A.~U.~Thor}

\maketitle

\abstracttitle{What is algebraic about algebraic effects and handlers?}
\abstractauthor[Andrej Bauer]{Andrej Bauer (University of Ljubljana, SI)}
\license

In this tutorial we reviewed the classic treatment of algebraic theories and
their models in the category of sets. We then drew an uninterrupted line of
thought from the classical theory to algebraic effects and handlers. First we
generalized operations with integral arities to parameterized operations with
arbitrary arities, as these are needed for modeling computational effects. The
free models of theories with generalized operations can be used as denotations
of effectful programs. The universal property of a free models can be used to
derive the notion of handlers. At the level of types, the value types correspond
to sets of generators and the computation types to the free models. The naive
set-theoretic treatment presented in the tutorial should be replaced with a
domain-theoretic one if we wanted adequate denotational semantics of a realistic
programming language with general recursion.


\abstracttitle{What is coalgebraic about algebraic effects and handlers?}
\abstractauthor[Matija Pretnar]{Matija Pretnar (University of Ljubljana, SI)}
\license

In this tutorial we reviewed the work of Gordon Plotkin and John
Power~\cite{plotkin08:_tensor_comod_model_operat_seman} in which they proposed
comodels of algebraic theories and tensoring of comodels and models as a
mathematical model for the interaction of an effectful program with its external
environment. A comodel of a theory $\mathsf{T}$ in a category $\mathbf{C}$ is a
model of~$\mathsf{T}$ in the opposite category $\mathbf{C}^{\mathrm{op}}$, and
the category of comodels in~$\mathbf{C}$ is the opposite of the category of
models in $\mathbf{C}^{\mathrm{op}}$. We may define the tensor $M \otimes W$ of
a comodel~$W$ and a model~$M$, which is a certain quotient of the product
$M \times W$. A pair $(p, w) \in M \times W$ may be viewed as a program~$p$
running in the external environment~$w$. The comodel~$W$ provides resources
needed for execution of algebraic operations in~$M$. In the tutorial we
emphasized the fact that comodels and tensoring are a more appropriate model of
top-level behavior of effectful program than various notions of ``top-level'' or
``default'' handlers. A handler has access to the continuation, but at the top
level this is not the case, or else programs would be able to control the
external world. It has to be the other way around.


\begin{thebibliography}{10}

\bibitem{plotkin08:_tensor_comod_model_operat_seman}
Gordon~D. Plotkin and John Power.
\newblock Tensors of comodels and models for operational semantics.
\newblock {\em Electronic Notes in Theoretical Computer Science}, 218:295--311,
  2008.

\end{thebibliography}


\end{document}